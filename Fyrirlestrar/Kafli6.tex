

\mode<presentation>
{
  \usetheme{boxes}
  %\useoutertheme{infolines}
  % með efnisyfirliti: Szeged, Frankfurt 
  % án efnisyfirlits: Pittsburgh
  % áhugavert: CambridgeUS, Boadilla
  %\setbeamercovered{transparent} %gegnsætt
  \setbeamercovered{invisible}

\defbeamertemplate*{footline}{infolines theme}
{
  \leavevmode%
  \hbox{%
  \begin{beamercolorbox}[wd=.333333\paperwidth,ht=2.25ex,dp=1ex,center]{author in head/foot}%
  %  \usebeamerfont{author in head/foot}\insertshortauthor~~\beamer@ifempty{\insertshortinstitute}{}{(\insertshortinstitute)}
  \end{beamercolorbox}%
  \begin{beamercolorbox}[wd=.333333\paperwidth,ht=2.25ex,dp=1ex,center]{title in head/foot}%
   % \usebeamerfont{title in head/foot}\insertshorttitle
  \end{beamercolorbox}%
  \begin{beamercolorbox}[wd=.333333\paperwidth,ht=2.25ex,dp=1ex,right]{date in head/foot}%
    %\usebeamerfont{date in head/foot}\insertshortdate{}\hspace*{2em}
    \insertshortlecture.\insertframenumber{} / \insertshortlecture.\inserttotalframenumber\hspace*{2ex} 
  \end{beamercolorbox}}%
  \vskip0pt%
}
\resetcounteronoverlays{rtaskno} %Does not increase counter rtaskno on \pause in beamer


}


\usepackage[english,icelandic]{babel}
\usepackage[utf8]{inputenc}
\usepackage{t1enc}
\usepackage{graphicx}
\usepackage{amsmath}
\usepackage{amssymb}
\usepackage{mathrsfs}
\usepackage{verbatim}
\usepackage{esint}


% RAGNAR SIGURÐSSON
%\usepackage[T1]{fontenc} 
%\usepackage[icelandic]{babel}
\usepackage{latexsym,amssymb,amsmath}
%\usepackage[utf8]{inputenc}
%\usepackage{graphicx}
\usepackage{epstopdf}
\usepackage{verbatim}
\usepackage{array,tabularx,arydshln}
\setbeamertemplate{theorems}[numbered]


\newtheorem{setning}{Setning}
\newtheorem{hjalpar}{Hjálparsetning}
\theoremstyle{definition}
\newtheorem{rithattur}{Ritháttur}
\newtheorem{skilgreining}{Skilgreining}
\newtheorem{daemi}{Dæmi}
\newtheorem{ath}{Athugasemd}

\newcommand\Wider[2][3em]{%
\makebox[\linewidth][c]{%
  \begin{minipage}{\dimexpr\textwidth+#1\relax}
  \raggedright#2
  \end{minipage}%
  }%
}

%counter used for blocks
\newcounter{rtaskno}
\DeclareRobustCommand{\rtask}[1]{%
   \refstepcounter{rtaskno}%
   \kaflanr.\thertaskno\label{#1}}

\newcommand{\C}{{\mathbb  C}}
\newcommand{\Z}{{\mathbb Z}}
\newcommand{\R}{{\mathbb  R}}
\newcommand{\N}{{\mathbb  N}}
\newcommand{\Q}{{\mathbb Q}}
\renewcommand{\phi}{\varphi}
\renewcommand{\epsilon}{\varepsilon}
\newcommand{\p}{{\partial}}
\renewcommand{\d}{{\partial}}

% RAGNAR SIGURÐSSON
\newcommand{\nin}{\mbox{$\;\not\in\;$}}
\newcommand{\dive}{\mbox{${\rm\bf div\,}$}}
\newcommand{\curl}{\mbox{${\rm\bf curl\,}$}}
\newcommand{\grad}{\mbox{${\rm\bf grad\,}$}}
\newcommand{\spann}{\mbox{${\rm Span}$}}
\newcommand{\tr}{\mbox{${\rm tr}$}}
\newcommand{\rank}{\mbox{${\rm rank}$}}
\newcommand{\image}{\mbox{${\rm image}$}}
\newcommand{\nullity}{\mbox{${\rm null}$}}
\newcommand{\proj}{\mbox{${\rm proj}$}}
\newcommand{\id}{\mbox{${\rm id}$}}
%\newcommand{\R}{\mbox{${\bf R}$}}
%\newcommand{\C}{\mbox{${\bf C}$}}
\newcommand{\Rn}{\mbox{${\bf R}^n$}}
\newcommand{\Rm}{\mbox{${\bf R}^m$}}
\newcommand{\Rk}{\mbox{${\bf R}^k$}}
\newcommand{\Av}{\mbox{${\bf A}$}}
\newcommand{\av}{\mbox{${\bf a}$}}
\newcommand{\uv}{\mbox{${\bf u}$}}
\newcommand{\vv}{\mbox{${\bf v}$}}
\newcommand{\wv}{\mbox{${\bf w}$}}
\newcommand{\xv}{\mbox{${\bf x}$}}
\newcommand{\zv}{\mbox{${\bf z}$}}
\newcommand{\yv}{\mbox{${\bf y}$}}
\newcommand{\bv}{\mbox{${\bf b}$}}
\newcommand{\cv}{\mbox{${\bf c}$}}
\newcommand{\dv}{\mbox{${\bf d}$}}
\newcommand{\ev}{\mbox{${\bf e}$}}
\newcommand{\fv}{\mbox{${\bf f}$}}
\newcommand{\gv}{\mbox{${\bf g}$}}
\newcommand{\hv}{\mbox{${\bf h}$}}
\newcommand{\iv}{\mbox{${\bf i}$}}
\newcommand{\jv}{\mbox{${\bf j}$}}
\newcommand{\kv}{\mbox{${\bf k}$}}
\newcommand{\pv}{\mbox{${\bf p}$}}
\newcommand{\nv}{\mbox{${\bf n}$}}
\newcommand{\qv}{\mbox{${\bf q}$}}
\newcommand{\rv}{\mbox{${\bf r}$}}
\newcommand{\sv}{\mbox{${\bf s}$}}
\newcommand{\tv}{\mbox{${\bf t}$}}
\newcommand{\ov}{\mbox{${\bf 0}$}}
\newcommand{\Fv}{\mbox{${\bf F}$}}
\newcommand{\Gv}{\mbox{${\bf G}$}}
\newcommand{\Uv}{\mbox{${\bf U}$}}
\newcommand{\Nv}{\mbox{${\bf N}$}}
\newcommand{\Hv}{\mbox{${\bf H}$}}
\newcommand{\Ev}{\mbox{${\bf E}$}}
\newcommand{\Sv}{\mbox{${\bf S}$}}
\newcommand{\Tv}{\mbox{${\bf T}$}}
\newcommand{\Bv}{\mbox{${\bf B}$}}
\newcommand{\Oa}{\mbox{$(0,0)$}}
\newcommand{\Ob}{\mbox{$(0,0,0)$}}
\newcommand{\Onv}{\mbox{$[0,0,\ldots,0]$}}
\newcommand{\an}{\mbox{$(a_1,a_2, \ldots,a_n)$}}
\newcommand{\xn}{\mbox{$(x_1,x_2, \ldots,x_n)$}}
\newcommand{\xnv}{\mbox{$[x_1,x_2, \ldots,x_n]$}}
\newcommand{\vnv}{\mbox{$[v_1,v_2, \ldots,v_n]$}}
\newcommand{\wnv}{\mbox{$[w_1,w_2, \ldots,w_n]$}}
\newcommand{\tvint}{\int\!\!\!\int}
\newcommand{\thrint}{\int\!\!\!\int\!\!\!\int}
\renewcommand{\ast}{{\operatorname{\text{astand}}}}


\usepackage{caption}
%\usepackage{pgfpages}
% \pgfpagesuselayout{2 on 1}[a4paper,border shrink=5mm]

\def\lecturename{Stærðfræðigreining IIB}
\title{\insertlecture}
\author{Sigurður Örn Stefánsson, \href{mailto:sigurdur@hi.is}{sigurdur@hi.is}}
\institute
{
  Verkfræði- og náttúruvísindasvið\\
  Háskóli Íslands
}
\subtitle{Stærðfræðigreining IIB, STÆ205G}
%\subject{\lecturename}

\mode<article>
{
	\usepackage[colorlinks=false,
	pdfauthor={Sigurður Örn Stefánson},
	%pdftitle={Töluleg greining}
	]{hyperref}
  %\usepackage{times}
  %\usepackage{mathptmx}
  \usepackage[left=1.5cm,right=4cm,top=1.5cm,bottom=3cm]{geometry}
}

% Beamer version theme settings

%\useoutertheme[height=0pt,width=2cm,right]{sidebar}
%\usecolortheme{rose,sidebartab}
%\useinnertheme{circles}
%\usefonttheme[only large]{structurebold}


\setbeamercolor{sidebar right}{bg=black!15}
\setbeamercolor{structure}{fg=blue}
\setbeamercolor{author}{parent=structure}

\setbeamerfont{title}{series=\normalfont,size=\LARGE}
\setbeamerfont{title in sidebar}{series=\bfseries}
\setbeamerfont{author in sidebar}{series=\bfseries}
\setbeamerfont*{item}{series=}
\setbeamerfont{frametitle}{size=}
\setbeamerfont{block title}{size=\small}
\setbeamerfont{subtitle}{size=\normalsize,series=\normalfont}


\setbeamertemplate{sidebar right}
{
  {\usebeamerfont{title in sidebar}%
    \vskip1.5em%
    \hskip3pt%
    \usebeamercolor[fg]{title in sidebar}%
    \insertshorttitle[width=2cm-6pt,center,respectlinebreaks]\par%
    \vskip1.25em%
  }%
  {%
    \hskip3pt%
    \usebeamercolor[fg]{author in sidebar}%
    \usebeamerfont{author in sidebar}%
    \insertshortauthor[width=2cm-2pt,center,respectlinebreaks]\par%
    \vskip1.25em%
  }%
  \hbox to2cm{\hss\insertlogo\hss}
  \vskip1.25em%
  \insertverticalnavigation{2cm}%
  \vfill
  \hbox to 2cm{\hfill\usebeamerfont{subsection in
      sidebar}\strut\usebeamercolor[fg]{subsection in
      sidebar}\insertshortlecture.\insertframenumber\hskip5pt}%
  \vskip3pt%
}%

\setbeamertemplate{title page}
{
  \vbox{}
  \vskip1em
  %{\huge Kapitel \insertshortlecture\par}
  {\usebeamercolor[fg]{title}\usebeamerfont{title}\inserttitle\par}%
  \ifx\insertsubtitle\@empty%
  \else%
    \vskip0.25em%
    {\usebeamerfont{subtitle}\usebeamercolor[fg]{subtitle}\insertsubtitle\par}%
  \fi%     
  \vskip1em\par
  %Vorlesung \emph{\lecturename}\ vom 
  \insertdate\par
  \vskip0pt plus1filll
  \leftskip=0pt plus1fill\insertauthor\par
  \insertinstitute\vskip1em
}

%\logo{\includegraphics[width=2cm]{beamerexample-lecture-logo.pdf}}



% Article version layout settings

\mode<article>

\makeatletter
\def\@listI{\leftmargin\leftmargini
  \parsep 0pt
  \topsep 5\p@   \@plus3\p@ \@minus5\p@
  \itemsep0pt}
\let\@listi=\@listI


\setbeamertemplate{frametitle}{\paragraph*{\insertframetitle\
    \ \small\insertframesubtitle}\ \par
}
\setbeamertemplate{frame end}{%
  \marginpar{\scriptsize\hbox to 1cm{\sffamily%
      \hfill\strut\insertshortlecture.\insertframenumber}\hrule height .2pt}}
\setlength{\marginparwidth}{1cm}
\setlength{\marginparsep}{1.5cm}

\def\@maketitle{\makechapter}

\def\makechapter{
  \newpage
  \null
  \vskip 2em%
  {%
    \parindent=0pt
    \raggedright
    \sffamily
    \vskip8pt
    %{\fontsize{36pt}{36pt}\selectfont Kapitel \insertshortlecture \par\vskip2pt}
    {\fontsize{24pt}{28pt}\selectfont \color{blue!50!black} \insertlecture\par\vskip4pt}
    {\Large\selectfont \color{blue!50!black} \insertsubtitle, \@date\par}
    \vskip10pt

    \normalsize\selectfont \@author\par\vskip1.5em
    %\hfill BLABLA
  }
  \par
  \vskip 1.5em%
}

\let\origstartsection=\@startsection
\def\@startsection#1#2#3#4#5#6{%
  \origstartsection{#1}{#2}{#3}{#4}{#5}{#6\normalfont\sffamily\color{blue!50!black}\selectfont}}

\makeatother

\mode
<all>



% Typesetting Listings

\usepackage{listings}
\lstset{language=Java}

\alt<presentation>
{\lstset{%
  basicstyle=\footnotesize\ttfamily,
  commentstyle=\slshape\color{green!50!black},
  keywordstyle=\bfseries\color{blue!50!black},
  identifierstyle=\color{blue},
  stringstyle=\color{orange},
  escapechar=\#,
  emphstyle=\color{red}}
}
{
  \lstset{%
    basicstyle=\ttfamily,
    keywordstyle=\bfseries,
    commentstyle=\itshape,
    escapechar=\#,
    emphstyle=\bfseries\color{red}
  }
}



% Common theorem-like environments

\theoremstyle{definition}
\newtheorem{exercise}[theorem]{\translate{Exercise}}




% New useful definitions:

\newbox\mytempbox
\newdimen\mytempdimen

\newcommand\includegraphicscopyright[3][]{%
  \leavevmode\vbox{\vskip3pt\raggedright\setbox\mytempbox=\hbox{\includegraphics[#1]{#2}}%
    \mytempdimen=\wd\mytempbox\box\mytempbox\par\vskip1pt%
    \fontsize{3}{3.5}\selectfont{\color{black!25}{\vbox{\hsize=\mytempdimen#3}}}\vskip3pt%
}}

\newenvironment{colortabular}[1]{\medskip\rowcolors[]{1}{blue!20}{blue!10}\tabular{#1}\rowcolor{blue!40}}{\endtabular\medskip}

\def\equad{\leavevmode\hbox{}\quad}

\newenvironment{greencolortabular}[1]
{\medskip\rowcolors[]{1}{green!50!black!20}{green!50!black!10}%
  \tabular{#1}\rowcolor{green!50!black!40}}%
{\endtabular\medskip}



\begin{document}

\section{Vigurreikningur}

\subsection{grad, div og curl} 

\subsubsection{Skilgreining \rtask{}}
 Skilgreinum {\em nabla}-virkjann sem diffurvirkja
$$\nabla=\iv\,\frac{\partial}{\partial x}+\jv\,\frac{\partial}{\partial y}+\kv\,\frac{\partial}{\partial z}.$$



\subsubsection{Skilgreining \rtask{}}
 Látum
$\Fv(x,y,z)=F_1(x,y,z)\,\iv+F_2(x,y,z)\,\jv+F_3(x,y,z)\,\kv$ vera
vigursvið og $\phi(x,y,z)$ vera fall. 

Skilgreinum  {\em stigul} $\phi$ sem vigursviðið 
$$\grad \phi=\nabla\phi=\frac{\partial \phi}{\partial x}\,\iv+
\frac{\partial \phi}{\partial y}\,\jv+\frac{\partial \phi}{\partial z}\,\kv.$$

Skilgreinum {\em sundurleitni} (e.~divergens) vigursviðsins $\Fv$ sem 
 $$\dive\Fv=\nabla\cdot\Fv=\frac{\partial F_1}{\partial x}+\frac{\partial F_2}{\partial y}+\frac{\partial F_3}{\partial z}.$$

Skilgreinum {\em rót}  vigursviðsins $\Fv$ sem 
 \begin{align*}
 \curl\Fv&=\nabla\times\Fv =\begin{vmatrix} \iv&\jv&\kv\\
 \frac{\partial} {\partial x}&\frac{\partial}{\partial y}&\frac{\partial}{\partial z}\\F_1&F_2&F_3\end{vmatrix} \\ &=\bigg(\frac{\partial F_3}{\partial y}-\frac{\partial F_2}{\partial z}\bigg)\,\iv+\bigg(\frac{\partial F_1}{\partial z}-\frac{\partial F_3}{\partial x}\bigg)\,\jv+\bigg(\frac{\partial F_2}{\partial x}-\frac{\partial F_1}{\partial y}\bigg)\,\kv. 
 \end{align*}



\subsubsection{Varúð \rtask{}}
  Ef $\phi(x,y,z)$ er fall þá er $\nabla \phi(x,y,z)$ 
stigullinn af $\phi(x,y,z)$ en $\phi(x,y,z)\nabla$ er diffurvirki.



\subsubsection{Varúð \rtask{}}
 Sundurleitnin $\dive\Fv$ er fall $\R^3\rightarrow\R$ en rótið $\curl\Fv$ er vigursvið $\R^3\rightarrow\R^3$.



\subsubsection{Skilgreining \rtask{}}
 Látum
$\Fv(x,y)=F_1(x,y)\,\iv+F_2(x,y)\,\jv$ vera vigursvið.  Skilgreinum
{\em sundurleitni} $\Fv$ sem  
$$\dive\Fv=\nabla\cdot\Fv=\frac{\partial F_1}{\partial
  x}+\frac{\partial F_2}{\partial y}.$$ 
og {\em rót} $\Fv$ skilgreinum við sem 
$$\curl\Fv=\bigg(\frac{\partial F_2}{\partial x}-\frac{\partial
  F_1}{\partial y}\bigg)\,\kv.$$ 




\subsubsection{Reiknireglur \rtask{}}
  Gerum ráð fyrir að $\Fv$ og $\Gv$ séu
vigursvið og $\phi$ og $\psi$ föll.  Gerum ráð fyrir að þær
hlutafleiður sem við þurfum að nota séu skilgreindar og samfelldar. 

(a) $\nabla(\phi\psi)=\phi\nabla\psi+\psi\nabla\phi$.

(b)  $\nabla\cdot(\phi\Fv)=(\nabla\phi)\cdot\Fv+\phi(\nabla\cdot\Fv)$.

(c) $\nabla\times(\phi\Fv)=(\nabla\phi)\times\Fv+\phi(\nabla\times\Fv)$. 

(d)  $\nabla\cdot(\Fv\times\Gv)=(\nabla\times\Fv)\cdot\Gv
-\Fv\cdot(\nabla\times\Gv)$.

(e) $\nabla\times(\Fv\times\Gv)=(\nabla\cdot\Gv)\Fv
+(\Gv\cdot\nabla)\Fv-(\nabla\cdot\Fv)\Gv-(\Fv\cdot\nabla)\Gv$.

(f) $\nabla(\Fv\cdot\Gv)=\Fv\times(\nabla\times \Gv)+\Gv\times(\nabla\times \Fv)+(\Fv\cdot\nabla)\Gv+(\Gv\cdot\nabla)\Fv$.

(g) $\nabla\cdot(\nabla\times \Fv)=0\qquad\qquad\dive\curl=0$

(h) $\nabla\times(\nabla\phi)=\ov\qquad\qquad\curl\grad=\ov$

(i)  $\nabla\times(\nabla\times \Fv)=\nabla(\nabla\cdot\Fv)-\nabla^2\Fv$.




\subsubsection{Skilgreining \rtask{}}
 Látum $\Fv$ vera vigursvið skilgreint á svæði $D$.  

(a) Vigursviðið $\Fv$ er sagt vera {\em sundurleitnilaust}
(e.~solenoidal) ef $\dive\Fv=0$ i öllum punktum  $D$.

(b) Vigursviðið $\Fv$ er sagt vera {\em rótlaust} (e.~irrotational) ef $\curl\Fv=\ov$ á öllu $D$.
    


\subsubsection{Athugasemd \rtask{}}
 Vigursvið   $\Fv(x,y,z)=F_1(x,y,z)\,\iv+F_2(x,y,z)\,\jv+F_3(x,y,z)\,\kv$ er rótlaust ef og aðeins ef 
$$\frac{\partial F_1}{\partial y}=
\frac{\partial F_2}{\partial x},\quad
\frac{\partial F_1}{\partial z}=
\frac{\partial F_3}{\partial x},\quad
\frac{\partial F_2}{\partial z}=
\frac{\partial F_3}{\partial y}.$$




\subsubsection{Setning \rtask{}}
 (a) Rót vigursviðs er sundurleitnilaus.

(b) Stigulsvið er rótlaust.

   \medskip



\subsubsection{Skilgreining \rtask{}}
  Svæði $D$ í rúmi eða plani kallast {\em
  stjörnusvæði} ef til er punktur $P$ í $D$ þannig að fyrir sérhvern
annan punkt $Q$ í $D$ þá liggur allt línustrikið á milli $P$ og $Q$ í
$D$. 




\subsubsection{Setning \rtask{}}
Látum $\Fv$ vera samfellt diffranlegt vigursvið
skilgreint á stjörnusvæði $D$.  Ef $\Fv$ er rótlaust þá er $\Fv$
stigulsvið.  Með öðrum orðum, ef vigursviðið  $\Fv$ er samfellt
diffranlegt og skilgreint á stjörnusvæði $D$ og uppfyllir jöfnurnar
$$\frac{\partial F_1}{\partial y}=
\frac{\partial F_2}{\partial x},\quad
\frac{\partial F_1}{\partial z}=
\frac{\partial F_3}{\partial x},\quad
\frac{\partial F_2}{\partial z}=
\frac{\partial F_3}{\partial y},$$
þá er $\Fv$ stigulsvið.


\subsubsection{Setning \rtask{}}
 Lát $\Fv$ vera samfellt diffranlegt vigursvið skilgreint á stjörnusvæði $D$.  Ef $\Fv$ er sundurleitnilaust þá er til vigursvið $\Gv$ þannig að $\Fv=\curl\Gv$.  Vigursviðið $\Gv$ kallast {\em vigurmætti} fyrir $\Fv$.




\subsection{Sundurleitnisetningin I} 

\subsubsection{Setning \rtask{} (Sundurleitnisetning I)}
 Látum $\Fv$ vera samfellt
diffranlegt vigursvið skilgreint á opnu mengi $D$ í $\R^3$.    Látum
$P$ vera punkt á skilgreiningarsvæði $\Fv$ og ${\cal S}_\epsilon$
kúluskel með miðju í $P$ og geisla $\epsilon$.  Látum svo $\Nv$ vera
einingarþvervigrasvið á ${\cal S}_\epsilon$ þannig að $\Nv$ vísar út á
við.  Þá er 
$$\dive \Fv(P)=\lim_{\epsilon\rightarrow 0^+}
\frac{1}{V_\epsilon}\tvint_{{\cal S}_\epsilon}\Fv\cdot\Nv\,dS.$$
þar sem $V_\epsilon = 4\pi\epsilon^3/3$ er rúmmálið innan í ${\cal S}_\epsilon$.



\subsubsection{Setning  \rtask{} (Setning Stokes I)}
  Látum $\Fv$ vera samfellt
diffranlegt vigursvið skilgreint á opnu mengi $D$ í $\R^3$.    Látum
$P$ vera punkt á skilgreiningarsvæði $\Fv$ og $C_\epsilon$ vera
hring með miðju í $P$ og geisla $\epsilon$.  Látum $\Nv$ vera
einingarþvervigur á planið sem hringurinn liggur í.  Áttum hringinn
jákvætt.
Þá er
$$\Nv\cdot\curl \Fv(P)=\lim_{\epsilon\rightarrow 0^+}
\frac{1}{A_\epsilon}\oint_{C_\epsilon}\Fv\cdot d\rv.$$
 þar sem $A_\epsilon = \pi\epsilon^2$ er flatarmálið sem afmarkast af  ${\cal C}_\epsilon$.



\subsubsection{Túlkun \rtask{}}
 Hugsum $\Fv$ sem lýsingu á vökvastreymi í $\R^3$.

$\dive \Fv(P)$ lýsir því hvort vökvinn er að þenjast út eða dragast
saman í punktinum $P$.  Sundurleitnisetningin (næsti fyrirlestur)
segir að samanlögð útþensla á rúmskika $R$ er jöfn streymi út um jaðar svæðisins $\mathcal{S}$,
eða 
$$\thrint_R\dive\Fv\,dV=\tvint_{\mathcal{S}} \Fv\cdot\Nv\,dS.$$

$\curl \Fv(P)$ lýsir hringstreymi í kringum punktinn $P$.  Setning
Stokes (þar næsti fyrirlestur) segir að samanlagt hringstreymi á fleti $\mathcal{S}$
er jafnt hringstreymi á jaðri flatarins, sem við táknum með $\mathcal{C}$, eða
$$\tvint_{\cal S} \curl\Fv\cdot\Nv\,dS=\oint_\mathcal{C} \Fv\cdot d\rv.$$



\subsubsection{Skilgreining \rtask{}}
 Látum $R$ vera svæði í $\R^2$ og $\cal C$
jaðar $R$.  Gerum ráð fyrir að $\cal C$ samanstandi af endanlega
mörgum ferlum ${\cal C}_1, \ldots, {\cal C}_n$.  Jákvæð áttun á
ferlunum felst í því að velja fyrir hvert $i$ stikun $\rv_i$ á ${\cal
  C}_i$ þannig að ef labbað eftir ${\cal C}_i$ í stefnu stikunar þá er
$R$ á vinstri hönd.




\subsubsection{Setning Green \rtask{}}
  Látum $R$ vera svæði í planinu þannig að
jaðar $R$, táknaður með $\cal C$,  
samanstendur af endanlega mörgum samfellt diffranlegum
ferlum.  Áttum $\cal C$ jákvætt.  Látum
$\Fv(x,y)=F_1(x,y)\,\iv+F_2(x,y)\,\jv$ vera samfellt diffranlegt
vigursvið skilgreint á $R$.  Þá er 
$$\oint_{\cal C}F_1(x,y)\,dx+F_2(x,y)\,dy=\tvint_R
\frac{\partial  F_2}{\partial x}- 
\frac{\partial  F_1}{\partial y}\,dA.$$



\subsubsection{Fylgisetning \rtask{}}
 Látum $R$ vera svæði í planinu þannig að
jaðar $R$ táknaður með $\cal C$, 
samanstendur af endanlega mörgum samfellt diffranlegum
ferlum.  Áttum $\cal C$ jákvætt. 
Þá er 
$$\mbox{Flatarmál } R=\oint_{\cal C}x\,dy= 
-\oint_{\cal C}y\,dx=\frac{1}{2}\oint_{\cal C}x\,dy-y\,dx.$$


\subsubsection{Sundurleitnisetningin í tveimur víddum \rtask{}}

Látum $R$ vera svæði í planinu þannig að
jaðar $R$, táknaður með $\cal C$,  
samanstendur af endanlega mörgum samfellt diffranlegum
ferlum.  Látum $\Nv$ tákna einingarþvervigrasvið á $\cal C$ þannig að
$\Nv$ vísar út úr $R$.  Látum
$\Fv(x,y)=F_1(x,y)\,\iv+F_2(x,y)\,\jv$ vera samfellt diffranlegt
vigursvið skilgreint á $R$.  Þá er 
$$\tvint_R\dive \Fv\,dA=\oint_{\cal C} \Fv\cdot\Nv\,ds.$$



\subsection{Sundurleitnisetningin II} 

\subsubsection{Skilgreining \rtask{}}
 Flötur er sagður reglulegur ef hann hefur snertiplan í hverjum punkti.  

Flötur $\cal S$ sem er búinn til með því að taka endanlega marga reglulega fleti ${\cal S}_1, \ldots, {\cal S}_n$ og líma þá saman á jöðrunum kallast {\em reglulegur á köflum}. 

\medskip
Þegar talað um einingarþvervigrasvið á slíkan flöt þá er átt við
vigursvið sem er skilgreint á fletinum nema í þeim punktum þar sem
fletir ${\cal S}_i$ og  ${\cal S}_j$ hafa verið límdir saman.  Í
slíkum punktum þarf flöturinn ekki að hafa snertiplan og því ekki
heldur þvervigur.

\medskip
Flötur er sagður {\em lokaður} ef hann er yfirborð svæðis í $\R^3$
(t.d. er kúluhvel lokaður flötur).



\subsubsection{Setning \rtask{} (Sundurleitnisetningin, Setning Gauss) }
 Látum $\cal S$ vera lokaðan flöt sem er reglulegur á köflum.  Táknum með $D$ rúmskikann sem $\cal S$ umlykur.  Látum $\Nv$ vera einingarþvervigrasvið á $\cal S$   sem vísar út úr $D$.  Ef $\Fv$ er samfellt diffranlegt vigursvið skilgreint á $D$ þá er 
$$\thrint_D \dive \Fv\,dV=\tvint_{\cal S} \Fv\cdot\Nv\,dS.$$



\subsubsection{Skilgreining \rtask{}}
 Látum $D$ vera rúmskika í $\R^3$.  Segjum að rúmskikinn $D$ sé $z$-{\em einfaldur} ef til er svæði $D_z$ í planinu og samfelld föll $f$ og $g$ skilgreind á $D_z$ þannig að 
$$D=\{(x,y,z)\mid (x,y)\in D_z\mbox{ og }f(x,y)\leq z\leq g(x,y)\}.$$
Það að rúmskiki sé $x$- eða $y$-einfaldur er skilgreint á sama hátt. 





\subsubsection{Setning \rtask{}}
Látum $\cal S$ vera lokaðan flöt sem er reglulegur á köflum.  Táknum með $D$ rúmskikann sem $\cal S$ umlykur.  Látum $\Nv$ vera einingarþvervigrasvið á $\cal S$   sem vísar út úr $D$.  Ef $\Fv$ er samfellt diffranlegt vigursvið skilgreint á $D$ og $\phi$ diffranlegt fall skilgreint á $D$ þá er
$$\thrint_D\curl \Fv\,dV=-\tvint_{\cal S}\Fv\times\Nv\,dS,$$
og 
$$\thrint_D\grad\phi\,dV=\tvint_{\cal S}\phi\Nv\,dS.$$

Athugið að útkomurnar úr heildunum eru vigrar.




\subsection{Setning Stokes} 

\subsubsection{Skilgreining \rtask{}}
Látum $\cal S$ vera áttanlegan flöt sem er
reglulegur á köflum með
jaðar $\cal C$ og einingarþver\-vigrasvið $\Nv$.  Áttun $\cal C$ út frá
$\Nv$ finnst með að hugsa sér að gengið sé eftir $\cal C$
þannig að skrokkurinn vísi í stefnu $\Nv$ og göngustefnan sé valin 
þannig að flöturinn sé á vinstri hönd. 



\subsubsection{Setning \rtask{} (Setning Stokes)}
Látum $\cal S$ vera áttanlegan flöt
sem er reglulegur á köflum og látum $\Nv$ tákna einingarþvervigrasvið
á $\cal S$.  Táknum með $\cal C$ jaðar $\cal S$ og
áttum $\cal C$ með tilliti til $\Nv$.    
Ef $\Fv$ er samfellt diffranlegt vigursvið
skilgreint á opnu mengi sem inniheldur $\cal S$  þá er  
$$\tvint_{\cal S} \curl\Fv\cdot\Nv\,dS=\oint_{\cal C}\Fv\cdot \Tv\,ds.$$



\subsubsection{Setning \rtask{}}
 Látum $\Fv$ vera samfellt
diffranlegt vigursvið skilgreint á opnu mengi $D$ í $\R^3$.    Látum
$P$ vera punkt á skilgreiningarsvæði $\Fv$ og $C_\epsilon$ vera
hring með miðju í $P$ og geisla $\epsilon$.  Látum $\Nv$ vera
einingarþvervigur á planið sem hringurinn liggur í.  Áttum hringinn
jákvætt.
Þá er
$$\Nv\cdot\curl \Fv(P)=\lim_{\epsilon\rightarrow 0^+}
\frac{1}{\pi\epsilon^2}\oint_{C_\epsilon}\Fv\cdot d\rv.$$
   


\subsubsection{Setning \rtask{}}
Látum $\cal S$ vera lokaðan flöt sem er
reglulegur á köflum.  Táknum með $D$ rúmskikann sem $\cal S$ umlykur.
Látum $\Nv$ vera einingarþvervigrasvið á $\cal S$   sem vísar út úr
$D$.  Ef $\Fv$ er samfellt diffranlegt vigursvið skilgreint á opnu
mengi sem inniheldur $D$, 
þá er 
$$\oint_{\cal S}\curl\Fv\cdot\Nv\,dS=0.$$



\subsection{Hagnýtingar í eðlisfræði} 

\subsubsection{Vökvaflæði \rtask{}}
Skoðum vökvaflæði í rúmi.  Hugsum okkur að vökvaflæðið sé líka háð tíma.  Látum $\vv(x,y,z,t)$ tákna hraðavigur agnar sem er í punktinum  $(x,y,z)$ á tíma $t$.  Látum $\delta(x,y,z,t)$ tákna efnisþéttleika (massi per rúmmálseiningu) í punktum $(x,y,z)$ á tíma $t$.  Þá gildir að 
$$\frac{\partial \delta}{\partial t}+\dive(\delta\vv)=0.$$
(Þessi jafna kallast samfelldnijafnan um vökvaflæðið.)


\subsubsection{Vökvaflæði \rtask{}}
 Til viðbótar við $\vv$ og $\delta$ þá skilgreinum við $p(x,y,z,t)$ sem þrýsting og $\Fv$ sem utanaðkomandi kraft, gefinn sem kraftur per massaeiningu.  Þá gildir að $$\delta\frac{\partial \vv}{\partial t}+\delta(\vv\cdot\nabla)\vv=-\nabla p+\delta\Fv.$$
(Þessi jafna er kölluð hreyfijafna flæðisins.)


\subsubsection{Rafsvið \rtask{} - Lögmál Coulombs }
 Látum punkthleðslu $q$ vera í punktinum $\sv=\xi\,\iv+\eta\,\jv+\zeta\,\kv$.   Í punktum $\rv=x\,\iv+y\,\jv+z\,\kv$ er rafsviðið vegna þessarar hleðslu
$$\Ev(\rv)=\frac{q}{4\pi\epsilon_0}\frac{\rv-\sv}{|\rv-\sv|^3}$$

þar sem $\epsilon_0$ er {\emph rafsvörunarstuðull} tómarúms.



\subsubsection{Rafsvið  \rtask{} - Lögmál Gauss (fyrsta jafna Maxwells)}
 Látum $\rho(\xi,\eta,\zeta)$ vera hleðsludreifingu og $\Ev$ rafsviðið vegna hennar.  Þá gildir að 
$$\dive\Ev=\frac{\rho}{\epsilon_0}.$$

\subsubsection{Rafsvið \rtask{}}
 Látum $\rho(\xi,\eta,\zeta)$ vera hleðsludreifingu á takmörkuðu svæði $R$ og $\Ev$ rafsviðið vegna hennar.  Ef við setjum
 $$ \phi(\rv) = -\frac{1}{4 \pi \epsilon_0} \iiint_R \frac{\rho(\sv)}{|\rv-\sv|} dV$$
 þá er $\Ev = \nabla \phi$ og þar með er 
$$\curl \Ev= \mathbf{0}.$$


 

\subsubsection{Segulsvið \rtask{} - Lögmál Biot-Savart }
 Látum straum $I$ fara eftir ferli $\cal F$.  Táknum segulsviðið með $\Hv$ og látum $\sv=\xi\,\iv+\eta\,\jv+\zeta\,\kv$ vera punkt á ferlinum $\cal F$.  Þá gefur örbútur $d\sv$ úr $\cal F$ af sér segulsvið 
$$d\Hv(\rv)=\frac{\mu_0 I}{4\pi}\frac{d\sv\times(\rv-\sv)}{|\rv-\sv|^3}$$

þar sem $\mu_0$ er {\emph segulsvörunarstuðull} tómarúms. Af þessu sést að 


$$\Hv=\frac{\mu_0 I}{4\pi}\oint_{\cal F}
\frac{d\sv\times(\rv-\sv)}{|\rv-\sv|^3}$$

og sýna má að ef $\rv \notin \mathcal{F}$ þá er $$\curl \Hv = \mathbf{0}.$$




\subsubsection{Segulsvið \rtask{} - Lögmál Ampére}
 Hugsum okkur að straumur $I$ fari upp eftir $z$-ás.  Táknum með $\Hv$ segulsviðið og $H=|\Hv|$.  Í punkti  $\rv=x\,\iv+y\,\jv+z\,\kv$ í fjarlægð $a$ frá $z$-ás er $H=\frac{\mu_0 I}{2\pi a}$ og ef $\cal C$ er lokaður einfaldur ferill sem fer rangsælis einu sinni umhverfis $z$-ásinn þá er 
$$\oint_{\cal C} \Hv\cdot d\rv=\mu_0 I.$$

 Hugsum okkur að $\mathbf{J}(\rv)$ sé straumþéttleiki í punkti $\rv$ (straumur á flatareiningu).  Þá er 
$$\curl \Hv = \mu_0 \mathbf{J}.$$

Einnig gildir að ef við setjum 
$$\Av(\rv)=\frac{\mu_0}{4\pi}\iiint_{R}
\frac{\mathbf{J}(\mathbf{s})}{|\rv-\sv|}dV,$$
þá er $\Hv=\curl \Av$  og því er $$\dive \Hv=0.$$


\subsubsection{Samantekt}
 \begin {align*}
  \dive \Ev &= \frac{\rho}{\epsilon_0} \quad~ \dive \Hv = 0 \\
  \curl \Ev &= \mathbf{0} \qquad \curl \Hv = \mu_0 \mathbf{J}
 \end {align*}

 Jöfnur Maxwells
 \begin {align*}
  \dive \Ev &= \frac{\rho}{\epsilon_0} \qquad ~ \dive \Hv = 0 \\
  \curl \Ev &= -\frac{\partial \Hv}{\partial t} \quad \curl \Hv = \mu_0 \mathbf{J} + \mu_0 \epsilon_0  \frac{\partial\Ev}{\partial t}
 \end {align*}





\end{document}
