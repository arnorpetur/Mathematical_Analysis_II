\def \kaflanr {21}
\lecture[\kaflanr]{\kaflanr. Flatarheildi. Áttanlegir fletir}{lecture-text}
\date{16.~mars 2015}
\newcounter{mycount}
\refstepcounter{mycount}

\begin{document}

\begin{frame}
	\maketitle
\end{frame}



\begin{frame}{Flatarheildi} 

\begin {block}{Verkefni  \rtask{}}
\begin {enumerate}
 \item Flatarmál flata -- sambærilegt við bogalengd ferla.  
\item Heildi falls yfir flöt með tilliti til flatarmáls -- sambærilegt við heildi falls eftir ferli með tilliti til bogalengdar.
\item Heildi vigursviðs yfir flöt -- svipar til heildis vigursviðs eftir ferli. 
 \end {enumerate}
\end{block}
\end{frame}



\begin{frame}{} 

\begin {block}{Skilgreining \rtask{}}
  Látum $\rv:D\rightarrow \R^3$ vera
reglulegan stikaflöt sem stikar flöt $\cal S$.  Flatarmál $\cal S$ er  
$$ A=\tvint_D\,dS=\tvint_D \big|{\textstyle\frac{\partial \rv}{\partial u}
\times\frac{\partial \rv}{\partial v}}\big|\,dudv.$$

\end{block}

\end{frame}

\begin{frame}{} 

\begin {block}{Formúla \rtask{}}
 Látum $f(x,y)$ vera diffranlegt fall skilgreint á
mengi $D$ í $\R^2$.  Flatarmál grafsins $z=f(x,y)$ er gefið með
formúlunni 
$$A=\tvint_D dS=\tvint_D {\textstyle\sqrt{1+
\big(\frac{\partial f}{\partial x}\big)^2+
\big(\frac{\partial f}{\partial y}\big)^2}}\,\,dx\,dy.$$
\end{block}

\end{frame}

\begin{frame}{} 

\begin {block}{Formúlur \rtask{form}}

 Ritum $dS$ fyrir flatarmálselement á fleti $\cal S$.  
\begin{itemize}
\item Ef $\rv:D\subseteq\R^2\rightarrow \R^3$ er stikun á $\cal S$ þá
  er $$dS=\bigg|\frac{\partial \rv}{\partial u}\times\frac{\partial
  \rv}{\partial v}\bigg|\,du\,dv.$$
\item Ef $\cal S$ er graf $z=g(x,y)$ þá er 
$$dS=\sqrt{1+g_1(x,y)^2+g_2(x,y)^2}\,dx\,dy.$$


\end{itemize}
\end{block}

\end{frame}

\begin{frame}
\begin {block}{Formúlur \kaflanr.\ref{form}, frh}

 Ritum $dS$ fyrir flatarmálselement á fleti $\cal S$.  
\begin{itemize}
\item Gerum ráð fyrir að flöturinn $\cal S$ í $\R^3$ hafi þann eiginleika að
  ofanvarp hans á $xy$-planið sé eintækt eða með öðrum orðum hægt er
  að lýsa fletinum sem grafi $z=f(x,y)$.
Ef $\nv$ er þvervigur á
flötinn og $\gamma$ er hornið sem þvervigurinn $\nv$ myndar við
jákvæða hluta $z$-ássins þá er 
$$dS=\bigg|\frac{1}{\cos\gamma}\bigg|\,dx\,dy
=\frac{|\nv|}{|\nv\cdot\kv|}\,dx\,dy.$$

Í þessu tilviki gildir einnig að ef $\cal S$ er lýst sem 
hæðarfleti $G(x,y,z)=C$ þá er 
$$dS=\bigg|\frac{\nabla G(x,y,z)}{G_3(x,y,z)}\bigg|\,dx\,dy.$$
\end{itemize}
\end{block}

\end{frame}





\begin{frame}{} 

\begin {block}{Skilgreining \rtask{}}
 Látum $\rv: D\rightarrow \R^3$ vera
reglulega stikun á fleti $\cal S$.   
Heildi falls $f(x,y,z)$ yfir flötinn $\cal S$ með tilliti til flatarmáls er
$$\tvint_{\cal S} f\,dS=\tvint_D f(\rv(u,v)) \big|{\textstyle\frac{\partial
    \rv}{\partial u} 
\times\frac{\partial \rv}{\partial v}}\big|\,dudv.$$
\end{block}

\end{frame}

\end{document}