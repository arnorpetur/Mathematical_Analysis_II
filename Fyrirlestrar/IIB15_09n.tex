\def \kaflanr {9}
\lecture[\kaflanr]{\kaflanr. Fólgin föll og Taylor-nálganir}{lecture-text}
\date{2.~febrúar 2015}
\newcounter{mycount}
\refstepcounter{mycount}

\begin{document}

\subsection{}
	\maketitle




\subsection{} 

\subsubsection{Upprifjun \kaflanr.\arabic{mycount}}\stepcounter{mycount}
 Skoðum feril sem gefinn er með jöfnu $F(x,y)=0$
og gerum ráð fyrir að báðar fyrsta stigs
hlutafleiður $F$ séu samfelldar.  Látum $(x_0,y_0)$
vera punkt á ferlinum.  Ef $F_2(x_0,y_0)\neq 0$ þá má skoða $y$ sem
fall af $x$ í grennd við punktinn $(x_0,y_0)$ og fallið $y=y(x)$ er
diffranlegt í punktinum $x_0$ og afleiðan er gefin með formúlunni
$$y'(x_0)=-\frac{F_1(x_0,y_0)}{F_2(x_0,y_0)}.$$
Sagt að jafnan $F(x,y)=0$ skilgreini $y$ sem {\em fólgið fall} af $x$
  í grennd við $(x_0,y_0)$.  






\subsection{} 

\subsubsection{Setning \kaflanr.\arabic{mycount}}\stepcounter{mycount}
Látum $F$ vera fall af $n$-breytum $x_1, \ldots,
x_n$ og gerum ráð fyrir að allar fyrsta stigs hlutafleiður  $F$ séu
samfelldar.   Látum $(a_1,\ldots,a_n)$ vera punkt þannig að 
$F(a_1,\ldots,a_n)=0$.  Ef $F_n(a_1,\ldots,a_n)\neq 0$ þá er til
samfellt diffranlegt fall $\varphi(x_1, \ldots, x_{n-1})$ skilgreint á 
opinni kúlu $B$
utan um $(a_1,\ldots,a_{n-1})$ þannig að 
$$\varphi(a_1,\ldots,a_{n-1})=a_n$$ 
og 
$$F(x_1,\ldots, x_{n-1}, \varphi(x_1, \ldots, x_{n-1}))=0$$
fyrir alla punkta $(x_1, \ldots, x_{n-1})$ í $B$.

Ennfremur gildir að 
$$\varphi_i(a_1,\ldots,a_{n-1})
=-\frac{F_i(a_1,\ldots,a_n)}{F_n(a_1,\ldots,a_n)}.$$ 






\subsection{} 

\subsubsection{Skilgreining \kaflanr.\arabic{mycount}}\stepcounter{mycount}
  
{\em Jacobi-ákveða} tveggja falla $u=u(x,y)$ og $v=v(x,y)$ með
tilliti til breytanna $x$ og $y$ er skilgreind sem 
$$\frac{\partial(u,v)}{\partial(x,y)}=
\begin{vmatrix} 
\frac{\partial u}{\partial x}&\frac{\partial u}{\partial y}\\
\noalign{\smallskip}
\frac{\partial v}{\partial x}&\frac{\partial v}{\partial y}
\end{vmatrix}.$$




\subsection{}

Ef $F$ og $G$ eru föll af breytum $x,y,z,\ldots$ þá skilgreinum við,
til dæmis,
$$\frac{\partial(F,G)}{\partial(x,y)}=
\begin{vmatrix} 
\frac{\partial F}{\partial x}&\frac{\partial F}{\partial y}\\
\noalign{\smallskip}
\frac{\partial G}{\partial x}&\frac{\partial G}{\partial y}
\end{vmatrix}\quad \mbox{og}\quad
\frac{\partial(F,G)}{\partial(y,z)}=
\begin{vmatrix} 
\frac{\partial F}{\partial y}&\frac{\partial F}{\partial z}\\
\noalign{\smallskip}
\frac{\partial G}{\partial y}&\frac{\partial G}{\partial z}
\end{vmatrix}.$$

Ef við höfum föll $F, G, H$ af breytum $x,y,z,w,\ldots$ þá
skilgreinum við, til dæmis,
$$\frac{\partial(F,G,H)}{\partial(w,z,y)}=
\begin{vmatrix} 
\frac{\partial F}{\partial w}&\frac{\partial F}{\partial z}
&\frac{\partial F}{\partial y}\\
\noalign{\smallskip}
\frac{\partial G}{\partial w}&\frac{\partial G}{\partial z}
&\frac{\partial G}{\partial y}\\
\noalign{\smallskip}
\frac{\partial H}{\partial w}&\frac{\partial H}{\partial z}
&\frac{\partial H}{\partial y}
\end{vmatrix}.$$





\subsection{} 

\subsubsection{Setning \kaflanr.\arabic{mycount}  (Upprifjun á reglu Cramers.)}\stepcounter{mycount}

 Látum $A$ vera andhverfanlegt
$n\times n$ fylki og $\bv$ vigur í $\Rn$.  Gerum ráð fyrir að
$\xv=(x_1, x_2, \ldots, x_n)$ sé lausn á $A\xv=\bv$.  Skilgreinum
$B_i$ sem $n\times n$ fylkið sem fæst með því að setja vigurinn $\bv$
í staðinn fyrir dálk $i$ í $A$.  Þá er
$$x_i=\frac{\det B_i}{\det A}.$$





\subsection{} 

\subsubsection{Setning \kaflanr.\arabic{mycount} (Setningin um fólgin föll)}\stepcounter{mycount}
Skoðum jöfnuhneppi
\begin{align*}
F_{(1)}(x_1,\ldots,x_m, y_1, \ldots, y_n)&=0\\
F_{(2)}(x_1,\ldots,x_m, y_1, \ldots, y_n)&=0\\
\vdots\\
F_{(n)}(x_1,\ldots,x_m, y_1, \ldots, y_n)&=0.
\end{align*}
Látum $P_0=(a_1,\ldots, a_m, b_1,\ldots, b_n)$ vera punkt sem uppfyllir
jöfnurnar.   
Gerum ráð fyrir að allar fyrsta stigs
hlutafleiður fallanna $F_{(1)},\ldots, F_{(n)}$ séu samfelldar á opinni kúlu umhverfis $P_0$ og að
$$\frac{\partial(F_{(1)}, \ldots, F_{(n)})}
{\partial( y_1, \ldots, y_n)}\,\bigg|_{P_0}\neq 0.$$


\subsection{}
\subsubsection{}
$\text{Þá eru til föll} \qquad \varphi_1(x_1,\ldots,x_m),\ldots,\varphi_n(x_1,\ldots,x_m)$ \\
á opinni kúlu $B$ umhverfis $(a_1,\ldots,a_m)$
þannig að 
$$\varphi_1(a_1,\ldots,a_m)=b_1,\ldots,\varphi_n(a_1,\ldots,a_m)=b_n \qquad \text{og}$$
\begin{align*}
F_{(1)}(x_1,\ldots,x_m, \varphi_1(x_1,\ldots,x_m),\ldots,
\varphi_n(x_1,\ldots,x_m)&=0\\
F_{(2)}(x_1,\ldots,x_m, \varphi_1(x_1,\ldots,x_m),\ldots,
\varphi_n(x_1,\ldots,x_m)&=0\\
\vdots\\
F_{(n)}(x_1,\ldots,x_m, \varphi_1(x_1,\ldots,x_m),\ldots,
\varphi_n(x_1,\ldots,x_m)&=0
\end{align*}
fyrir alla punkta $(x_1,\ldots,x_m)$ í $B$.
Enn fremur fæst að %ATH x_j í sæti i
$$\frac{\partial \varphi_i}{\partial x_j}
=\frac{\partial y_i}{\partial x_j}
=-\frac{\frac{\partial(F_{(1)}, \ldots, F_{(n)})}
{\partial( y_1, \ldots,x_j,\ldots y_n)}}
{\frac{\partial(F_{(1)}, \ldots, F_{(n)})}{\partial( y_1, \ldots, y_n)}}.$$






\subsection{} 

\subsubsection{Setning \kaflanr.\arabic{mycount} (Setningin um staðbundna andhverfu)}\stepcounter{mycount}
Látum $$\fv(x_1,\ldots,
x_n)=(f_1(x_1,\ldots,x_n),\ldots,f_n(x_1,\ldots,x_n))$$ vera vörpun af
$n$ breytistærðum sem tekur gildi í $\Rn$ og er skilgreind á opnu
mengi í $\Rn$.   Gerum ráð fyrir að allar
fyrsta stigs hlutafleiður fallanna $f_1, \ldots, f_n$ séu samfelld
föll.  Ef Jacobi-fylkið $D\fv(\xv_0)$ er andhverfanlegt í punkti
$\xv_0$ á skilgreiningarsvæði $\fv$ þá er til opin kúla $B_{\xv}$ utan
um $\xv_0$ og opin kúla $B_{\yv}$ utan um $\yv_0=f(\xv_0)$ og vörpun
$\gv:B_{\yv}\rightarrow B_{\xv}$ þannig að 
$\gv(\fv(\xv))=\xv$ fyrir alla punkta $\xv\in B_{\xv}$ og 
$\fv(\gv(\yv))=\yv$ fyrir alla punkta $\yv\in B_{\yv}$.






\subsection{} 

\subsubsection{Upprifjun \kaflanr.\arabic{mycount} (Taylor-regla í einni breytistærð.)}\stepcounter{mycount}
  Látum 
$f$ vera $n+1$-diffranlegt fall af einni breytistærð.  Margliðan 
$$P_{(n)}(x)=f(a)+f'(a)(x-a)+\frac{f''(a)}{2!}(x-a)^2
+\cdots+\frac{f^{(n)}(a)}{n!}(x-a)^n$$
kallast {\em $n$-ta stigs Taylor-margliða $f$ með miðju í $a$}.
Til er punktur $s$ á milli $a$ og $x$ þannig að 
$$E_{(n)}(x)=f(x)-P_{(n)}(x)=\frac{f^{(n+1)}(s)}{(n+1)!}(x-a)^{n+1}.$$
Fáum svo að 
\begin {align*}
&f(x)=P_{(n)}(x)+E_{(n)}(x) \\
&=f(a)+f'(a)(x-a)+\cdots+\frac{f^{(n)}(a)}{n!}(x-a)^n
+\frac{f^{(n+1)}(s)}{(n+1)!}(x-a)^{n+1}, 
\end {align*}

 
sem er kallað {\em $n$-ta stigs Taylor-formúla.}






\subsection{} 

\subsubsection{Skilgreining \kaflanr.\arabic{mycount}}\stepcounter{mycount}
 Látum $f(x,y)$ vera fall þannig að fyrsta stigs hlutafleiður $f$ eru skilgreindar og samfelldar.  Margliðan
$$P_{(1)}(x,y)=f(a,b)+f_1(a,b)(x-a)+f_2(a,b)(y-b)$$
kallast {\em fyrsta stigs Taylor-margliða $f$ með miðju í $(a,b)$}. 






\subsection{} 

\subsubsection{Skilgreining \kaflanr.\arabic{mycount}}\stepcounter{mycount}

Látum $f(x,y)$ vera fall þannig að fyrsta og annars
stigs hlutafleiður $f$ eru skilgreindar og samfelldar.  Margliðan
\begin{align*}P_{(2)}&(x,y)=f(a,b)+f_1(a,b)(x-a)+f_2(a,b)(y-b)\\
&+\frac{1}{2}\big(f_{11}(a,b)(x-a)^2+
2f_{12}(a,b)(x-a)(y-b)+f_{22}(a,b)(y-b)^2\big)
\end{align*}
kallast {\em annars stigs Taylor-margliða $f$ með miðju í $(a,b)$}.  






\subsection{} 

\subsubsection{Skilgreining og athugasemd \kaflanr.\arabic{mycount}}\stepcounter{mycount}
Skilgreinum tvo {\em diffurvirkja} $D_1$ og $D_2$ þannig að 
$$D_1f(a,b)=f_1(a,b)\qquad\mbox{og}\qquad
D_2f(a,b)=f_2(a,b).$$
Athugið að ef hlutafleiður $f$ af nógu háum stigum eru allar skilgreindar og samfelldar þá er $D_1D_2=D_2D_1$, þ.e.a.s.\ ekki skiptir máli í hvaða röð er diffrað, bara hve oft er diffrað með tilliti til hvorrar breytu.






\subsection{} 

\subsubsection{Upprifjun \kaflanr.\arabic{mycount}(Tvíliðuregla)}\stepcounter{mycount}
Skilgreinum 
$${n\choose j}=\frac{n!}{j!(n-j)!}.$$
Talan ${n\choose j}$ (lesið $n$ yfir $j$) er $j+1$ talan í $n+1$ línu Pascals-þríhyrningsins.
 Höfum að 
$$(x+y)^n=\sum_{j=0}^n \textstyle{n\choose j}x^jy^{n-j}.$$






\subsection{} 

\subsubsection{Regla \kaflanr.\arabic{mycount}}\stepcounter{mycount}

Ef $f(x,y)$ er fall þannig að allar hlutafleiður af $n$-ta og lægri stigum eru samfelldar þá gildir að 
$$(hD_1+kD_2)^nf(a,b)=\sum_{j=0}^n \textstyle{n\choose j}
h^jk^{n-j}D_1^jD_2^{n-j}f(a,b).$$






\subsection{} 

\subsubsection{Skilgreining \kaflanr.\arabic{mycount}}\stepcounter{mycount}
Fyrir fall $f(x,y)$ þannig að allar
hlutafleiður af $n$-ta og lægri stigum eru samfelldar þá er {\em $n$-ta
stigs Taylor-margliða $f$ með miðju í punktinum} $(a,b)$ skilgreind sem
margliðan  
\begin{align*}
P_{(n)}(x,y)&= \sum_{m=0}^n \frac{1}{m!}((x-a)D_1+(y-b)D_2)^m f(a,b)\\
&=\sum_{m=0}^n\sum_{j=0}^m \frac{1}{m!}\textstyle{m\choose j}
D_1^jD_2^{m-j}f(a,b)(x-a)^j(y-b)^{m-j}\\
&=\sum_{m=0}^n\sum_{j=0}^m \frac{1}{j!(m-j)!}
D_1^jD_2^{m-j}f(a,b)(x-a)^j(y-b)^{m-j}.
\end{align*}






\subsection{} 

\subsubsection{Setning \kaflanr.\arabic{mycount}}\stepcounter{mycount}
 Fyrir fall $f(x,y)$ þannig að allar hlutafleiður af $n+1$-ta og lægri stigum eru samfelldar þá gildir um skekkjuna í  $n$-ta stigs Taylor-nálgun að til er tala $\theta$ á milli 0 og 1 þannig að ef $h=x-a$ og $k=y-b$ þá er 
$$f(x,y)-P_{(n)}(x,y)=\frac{1}{(n+1)!}(hD_1+kD_2)^{n+1}
f(a+\theta h, b+\theta k).$$




\end{document}