\lecture[3]{Kafli 3: Línuleg algebra, Gauss-eyðing, \\
fylkjastaðll, skekkjumat, ástandstala,\\  
LU-þáttun og fastapunktsaðferðir}{lecture-text}
\date{2., 4.~og 9.~apríl 2014}

\begin{document}

\begin{frame}
	\maketitle
\end{frame}

%\section*{}
\begin{frame}{Yfirlit}
\begin{block}{Kafli 3: Jöfnuhneppi}
\begin{center}
\begin{tabular}{|l|l|l|l|}\hline
Kafli &Heiti á viðfangsefni &Bls. & Glærur\\
\hline
3.1 & Línuleg algebra & 149-159 & 3-5\\
3.2 & Vending (pivoting) & 160-170 & 6-9\\
3.3 & Fylkjastaðall (matrix norm) & 171-180 & 10-17\\
3.4 & Skekkjumat og ástandstala (condition numb.) & 181-190 & 18-23\\
3.5 & $LU$-þáttun & 191-204 & 24-39\\
3.8 & Fastapunktsaðferðir (fixed point iteration)& 223-236 & 40-49\\
3.10 & Newton-aðferð fyrir jöfnuhneppi &  249-258 & 50-55\\
3.8 & Fastapunktssetning fyrir jöfnuhneppi & & 56\\

\hline
\end{tabular}
\end{center}

\end{block}
\end{frame}

\section*{3.1 Línuleg algebra}
\begin{frame}{3.1 Línuleg jöfnuhneppi}
\begin{block}{Línuleg jöfnuhneppi}
 Gefið $n\times n$ fylki $A$ og $n$-vigur $\bv$ þá leitum við að
vigri $\xv$ þannig að 
$$
A\xv = \bv.
$$
\end{block}

\pause 

 \begin{block}{Lausnir}
  Við höfum almennt tvær leiðir til þess að leysa línuleg jöfnuhneppi: \pause
\begin{itemize}
 \item Gauss-eyðing og innsetning.\pause
\item Reikna andhverfu $A$, $A^{-1}$. Þá er 
$$ 
\xv = A^{-1}\bv.
$$
\end{itemize}
\end{block}
\end{frame}

\begin{frame}{3.1 Fjöldi aðgerða}
 \begin{block}{}
  \begin{itemize}
   \item Gauss-eyðing fyrir $n\times n$ fylki krefst $\frac 23n^3+\frac 12n^2 -\frac 76n$
reikniaðgerða. Innsetningin krefst svo $n^2$ aðgerða til viðbótar. 
\pause  Samanlagður fjöldi aðgerða er því
$$
\frac 23n^3+\frac 32n^2 -\frac 76n.
$$\pause
  \item Það að reikna $A^{-1}$ krefst hins vegar $2n^3-2n^2+n$ aðgerða og margföldunin 
$A^{-1}b$, krefst $2n^2-n$ aðgerða til viðbótar. 
\pause Samanlagður fjöldi aðgerða er því
$$
2n^3.
$$\pause
  \end{itemize}
Hér er greinilega gáfulegra að nota Gauss-eyðingu. Almennt
þá forðumst við eins og mögulegt er að reikna $A^{-1}$.
 \end{block}

\end{frame}

\begin{frame}{3.1 Vandamál með stöðugleika} 
\begin{block}{Einfalt dæmi}
 Skoðum jöfnuhneppið
$$
\left[\begin{array}{ll}
\epsilon & 1\\
1 & 1
\end{array}\right] \left[\begin{array}{l}
x_1\\
x_2
\end{array}\right] 
=\left[\begin{array}{l}
1\\
2
\end{array}\right]
$$
\pause
Nákvæm lausn er $x_1=1+\frac{\epsilon}{1-\epsilon}, 
x_2=1-\frac{\epsilon}{1-\epsilon}$. \pause

Ef hins vegar $\epsilon$ er minna en nákvæmnin í tölvunni sem erum að vinna á, þá 
gefur Gauss-eyðing í tölvu, þar sem eytt er með 1.~línu, svarið $x_1 = 0, x_2 = 1$. \pause

Ef línunum væri víxlað, þá gæfi tölvan hins vegar $x_1=1, x_2=1$ sem er miklu nær réttu svari.
Sjá nánar skrána {\tt tg14\_03synidaemi.pdf} á Uglu. 
\end{block}

\pause

\begin{block}{Athugasemd}
 Það er alveg ljóst að megum ekki framkvæma Gauss-eyðingu blindandi því þá getur
magnast upp styttingarskekkja sem skemmir lausnina okkar.
\end{block}
\end{frame}

\section*{3.2 Vending}

\begin{frame}{3.2 Vending (e. pivoting)}
 \begin{block}{Vandamálið}
  Það sem olli vandræðum í dæminu hér á undan var það að forystustuðull
fyrstu línurnar var hlutfallslega miklu minni en forystustuðull annarrar línu.
 \end{block}

\begin{block}{Lausnin}
 Lausnin felst í því að víxla á línum þannig að við þurfum ekki að notast
við litla forystustuðla.
\end{block}
\end{frame}

\begin{frame}{3.2 Hlutvending (e.~partial pivoting)}
\begin{block}{}
 Í grófum dráttum: Í umferð $i$ í Gauss-eyðingunni þá athugum við hvort tölugildi 
forystu\-stuðla línanna fyrir neðan
línu $i$ eru stærri en forystustuðull línu $i$, ef svo er þá víxlum við á 
þeirri línu og línu $i$.

\pause
\medskip

Það er, í $i$-tu ítrun Gauss-eyðingar þá látum við $M_i = \max_{i\leq j \leq n} |a_{ji}|$.
Ef $|a_{ii}| < M_i$ þá víxlum við á línu $i$ og fyrstu línunni fyrir neðan 
sem hefur forystustuðul með tölugildi jafnt og $M_i$. \pause (Þetta þýðir að ef 
$j_0 = \min\{ j ; i \leq j\leq n \text{ og } a_{ji} = M_i \}$ þá víxlum við á línu $i$ og $j_0$).
\end{block}

\pause
\begin{block}{Vankantar}
Hlutvending virkar oft vel en getur búið til skekkju þar sem hún tekur bara tillit til
forystustuðlanna í hverri línu, sjá dæmi kafla 3.2 (bls.~165). 
\end{block}
\end{frame}

\begin{frame}{3.2 Sköluð hlutvending (e.~scaled partial pivoting)}
 \begin{block}{}
  Skilgreinum vigurinn $\sv$ sem heldur utan um ``stærð`` línanna í $A$,
$$
s_i = \max_{1\leq j \leq n} |a_{ij}|.
$$
\pause
Látum dálkvigurinn $\rv$ halda utan um það hvernig við umröðum línunum í $A$.
Byrjum með
$$
\rv = [1\ 2\ 3\ 4\ \ldots\ n]^T.
$$
\end{block}
\pause


\begin{block}{Athugasemd} 
Við munum uppfæra $\rv$ eftir þörfum en breytum ekki $\sv$ (of dýrt ef $n$ stórt). 
\end{block}

\end{frame}

\begin{frame}{3.2 Sköluð hlutvending, framh.}
\begin{block}{}
Í ítrun $i$ þá látum við 
$$
M_i = \max_{i \leq j \leq n} \frac{|a_{{r_j}i}|}{s_{r_j}},
$$
og látum $j_0$ vera minnsta $j$ þannig að hámarkinu er náð, 
$$
\frac{ |a_{r_{j_0}i}|}{s_{r_{j_0}}} = M_i.
$$
\pause

Ef $i < j_0$ þá skiptum við á línum $i$ og $j_0$, \pause þ.e.
$$
\rv = [\ldots\ i\ \ldots\ j_0\ \ldots]^T \text{ breytist í }
\rv = [\ldots\ j_0\ \ldots\ i\ \ldots]^T.
$$
 \end{block}
\end{frame}

% Sýnidæmi

\section*{3.3 Fylkjastaðall}

\begin{frame}{3.3 Inngangur}
 \begin{block}{Að mæla fjarlægð milli hluta}
    Á rauntalnalínunni þá mælum við fjarlægð með tölugildinu,
    þannig að fjarlægðin á milli $x$ og $y$ er gefin með
    $d(x,y)=|x-y|$. 

\pause\smallskip

  Í $\R^n$ þá finnst okkur evklíðski staðallinn náttúrulegur, enda svarar  
  hann til þess að mæla fjarlægð milli punkta með beinni reglustiku;
  $$
    d(\xv,\yv) = \sqrt{ (x_1-y_1)^2 + \ldots + (x_n-y_n)^2 }.
  $$\pause
  Þetta er hins vegar ekki eina leiðin til þess mæla fjarlægð
í $\R^n$, eins og við sjáum fljótlega, og ekki endalega réttari 
en aðrar aðferðir.

\pause\smallskip
  
  Almennt viljum við geta mælt ''fjarlægð`` á milli allra þeirra hluta
sem við erum skoða, hvort sem það eru margliður, föll eða fylki.
Tilgangurinn er að geta metið hversu langt nálgunin okkar er frá
réttu gildi og hversu stór skekkjan er í samanburði við ''stærð``
hlutarins sem við erum að vinna með.

 \end{block}
\end{frame}


\begin{frame}{3.3 Vigurstaðall}
 
\begin{block}{Skilgreining}
 Fall $\| \cdot\|:\R^n \to \R$ kallast \emph{vigurstaðall} (e.~vector norm)
ef fyrir öll $\xv,\yv\in \R^n$ og $\alpha \in \R$ gildir eftirfarandi:\pause
\begin{enumerate}
 \item $\|\xv\| \geq 0$\pause
 \item $\|\xv\| = 0$ ef og aðeins ef $\xv = 0$\pause
 \item $\|\alpha\xv\| = |\alpha|\|\xv\|$\pause
 \item $\|\xv+\yv\| \leq \|\xv\| + \|\yv\|$
\end{enumerate}
\end{block}

\pause

\begin{block}{Athugasemd}
 Tölugildisfallið á $\R$ er greinilega staðall. 
\end{block}
\end{frame}

\begin{frame}{3.3 Dæmi um staðla}
\begin{block}{$\ell_2$ staðallinn}
 Einnig kallaður evklíðska fjarlægðin, er gefinn með 
$$
\|\xv\|_2 = \left( \sum_{j=1}^n x_j^2 \right)^{\frac 12} = \sqrt{\xv \cdot \xv}.
$$
\end{block}
\vspace{-0.5cm}
\pause

\begin{block}{$\ell_\infty$ staðallinn}
$$
  \|\xv\|_\infty = \max_{1\leq j \leq n} |x_j|.
$$
\end{block}
\vspace{-0.5cm}
\pause

\begin{block}{$\ell_p$ staðlar}
 Almennt, ef $1\leq p < \infty$, þá skilgreinum við 
$$
\|\xv\|_p = \left( \sum_{j=1}^n |x_j|^p \right)^{\frac 1p}.
$$
\end{block}

\end{frame}

\begin{frame}{3.3 Fylkjastaðall}
 \begin{block}{Skilgreining}
  \emph{Fylkjastaðall} (e.~matrix norm) er fall $\|\cdot\|:\R^{n\times n} \to \R$,
þannig að fyrir öll $A,B \in \R^{n\times n}$ og $\alpha \in \R$ gildir\pause
\begin{enumerate}
 \item $\|A\| \geq 0$\pause
\item $\|A\| = 0$ ef og aðeins ef $A=0$\pause
\item $\| \alpha A\| = |\alpha|\|A\|$\pause
\item $\|A+B\| \leq \|A\| + \|B\|$ \pause
\item $\|AB\| \leq \|A\|\|B\|$
\end{enumerate}
 \end{block}

\begin{block}{Athugasemd}
 Ef þessi skilgreining er borin saman við skilgreininguna á staðli fyrir vigurrúm þá 
sjáum við að eini raunverulegi munurinn er skilyrði 5.
\end{block}


\end{frame}

\begin{frame}{3.3 Fylkjastaðall skilgreindur út frá vigurstaðli}
 \begin{block}{Skilgreining}
  Látum $\|\cdot\|$ vera vigurstaðal. Fallið $\|\cdot\|:\R^{n\times n} \to \R$
sem skilgreint er með 
$$
\|A\| = \max_{\|\xv\| \neq 0} \frac{\|A\xv\|}{\|\xv\|},
$$
kallast \emph{náttúrulegi fylkjastaðallinn} sem $\|\cdot\|$ gefur af sér.
 \end{block}

\pause

\begin{block}{Athugasemd}
 Það þarf að sýna, og er ekki mjög erfitt, að þessi fylkjastaðall uppfyllir
öll skilyrðin í skilgreiningu hér á undan og er því sannarlega fylkjastaðall.
\end{block}

\pause

\begin{block}{Athugasemd}
 Ef $\|\cdot\|$ er náttúrulegur fylkjastaðall þá gildir að fyrir öll fylki $A$ og alla vigra $\xv$ að
$$
\underbrace{\|A\xv\|}_{\text{vigurstaðll}} \leq \underbrace{\|A\|}_{\text{fylkjastaðall}}
\underbrace{\|\xv\|}_{\text{vigurstaðall}}.
$$
\end{block}


\end{frame}

\begin{frame}{3.3 Dæmi um fylkjastaðal}
\begin{block}{Athugasemd}
Fyrir sérhvern $\ell_p$ staðal fáum við fylkjastaðal $\|\cdot\|_p$. 
\end{block}

\begin{block}{$\|\cdot\|_\infty$}
Einfaldastur er staðallinn sem tilheyrir $\ell_\infty$, en hann uppfyllir
$$
\|A\|_\infty = \max_{1\leq i \leq n} \sum_{j=1}^n |a_{ij}|.
$$
\end{block}
\end{frame}

\begin{frame}{3.3 Eigingildi}
\begin{block}{Skilgreining}
 Látum $A$ vera fylki. Ef tala $\lambda$ (hugsanlega tvinntala) og vigur $\xv$ uppfylla 
$$
A\xv = \lambda \xv,
$$
þá kallast $\lambda$ \emph{eigingildi} $A$, og $\xv$ \emph{eiginvigur} $A$.\pause

\medskip

Athugið að eigingildi $A$ eru nákvæmlega rætur kennimargliðu $A$, $t \mapsto \det(A-It)$. 
\end{block}
\end{frame}

\begin{frame}{3.3 Róf og eiginleikar þess}
\begin{block}{Skilgreining}
Mengi allra eigingilda $A$ er kallað \emph{róf $A$} (e.~spectrum) og er táknað með $\sigma(A)$.
\pause

\emph{Rófgeisli} (e.~spectral radius) fylkisins $A$ er talan
$$
\rho(A) = \max_{\lambda \in \sigma(A)} |\lambda|.
$$
\end{block}


\begin{block}{Setning}
Látum $A$ vera fylki, þá gildir eftirfarandi \pause
 \begin{itemize}
  \item $\|A\|_2 = \sqrt{\rho(A^T A)}$\pause
  \item $\rho(A) \leq \|A\|$ fyrir sérhvern náttúrulegan fylkjastaðal $\|\cdot\|$\pause
  \item Fyrir sérhvert $\epsilon >0$ þá er til náttúrulegur fylkjastaðall $\|\cdot\|$ þannig
að $\|A\| \leq \rho(A) + \epsilon$.
 \end{itemize}

\end{block} 
\end{frame}

\section*{3.4 Skekkjumat og ástandstala}
\begin{frame}{3.4 Hvernig á að mæla skekkju}
 Gerum ráð fyrir að $A$ sé andhverfanlegt fylki, $\bv$ einhver vigur og að við séum að
leita að lausn $\xv$ á 
$$
A\xv = \bv.
$$
\pause
\medskip

Ef við höfum nálgun $\tilde \xv$ þannig að \emph{leifin} (e.~residual) $\rv = A\tilde \xv - \bv$ er lítil, hvað 
getum við þá sagt um \emph{skekkjuna} (e.~error) $\ev = \xv -\tilde \xv$? Er hún endilega lítil?

\pause
\medskip

Sjáum að svo er ekki, skekkjan getur verið hlutfallslega miklu meiri heldur en leifin
(Example 3.11).


\end{frame}

\begin{frame}{3.4 Skekkjumat}
Við höfum fjórar jöfnur
$$
\bv=A\xv, \quad \xv=A^{-1}\bv, \quad 
\rv=A(\mathbf{\tilde x}-\xv)=A\ev, \quad \text{ og } \quad
\ev=A^{-1}\rv
$$\pause
og þær gefa okkur fjórar ójöfnur fyrir tilsvarandi staðal:
$$
\|\bv\|\leq \|A\|\|\xv\|, \quad  \|\xv\|\leq \|A^{-1}\|\|\bv\|, 
\quad  
\|\rv\|\leq \|A\|\|\ev\|, \quad  \|\ev\|\leq \|A^{-1}\|\|\rv\|
$$

\pause
\smallskip
Við getum tengt tvær síðustu ójöfnurnar saman í mat á skekkjunni
$$
\dfrac 1{\|A\|}\cdot  \|\rv\|
\leq  \|\ev\| \leq 
\|A^{-1}\| \|\rv\|
$$\pause
og með því að nota fyrstu tvær ójöfnurnar fæst mat á hlutfallslegri skekkju
$$
\dfrac 1{\|A\|\|A^{-1}\|}\dfrac{ \|\rv\|}{\|\bv\|} 
\leq \dfrac{\|\ev\|}{\|\xv\|} \leq 
\|A\|\|A^{-1}\|\dfrac{\|\rv\|}{\|\bv\|}
$$

\pause
\smallskip
Nú skilgreinum við {\it ástandstölu fylkisins} $A$ með
$$
\kappa(A)=\|A\|\|A^{-1}\|.
$$
\end{frame}


\begin{frame}{3.4 Ástandstala fylkis og mat á hlutfallslegri skekkju} 

\begin{block}{Matið}
Með ástandstölunni verður mat okkar á hlutfallslegu skekkjunum að
$$
\dfrac 1{\kappa(A)}\cdot \dfrac{ \|\rv\|}{\|\bv\|} 
\leq \dfrac{\|\ev\|}{\|\xv\|} \leq 
\kappa(A)\cdot \dfrac{\|\rv\|}{\|\bv\|}
$$
\end{block}

\pause

\begin{block}{Athugasemd}
Athugið að skilgreiningin 
$$
\kappa(A)=\|A\|\|A^{-1}\|.
$$
er {\it mjög}  háð því hvaða staðal við veljum,
\pause
en við höfum þó að
$$
1=\|I\|=\|AA^{-1}\|\leq \|A\|\|A^{-1}\|=\kappa(A) 
$$
\end{block}
\end{frame}

\begin{frame}{3.4 Áhrif gagnaskekkju} 

Hugsum okkur nú að við viljum leysa jöfnuhneppi $A\xv=\bv$, en vegna
skekkju í stuðlum jöfnuhneppisins  leysum við annað hneppi
$\tilde A\mathbf {\tilde x}=\mathbf{\tilde b}$.

\pause
\smallskip
Við skilgreinum {\it gagnaskekkjur} $\delta A=\tilde A-A$ og
$\delta\bv=\mathbf{\tilde b}-\bv$ og ætlum að nota þær til þess að
meta skekkjuna $\ev=\delta\xv=\mathbf{\tilde x}-\xv$.

\pause
\smallskip
Við verðum að gera ráð fyrir að $ \|\delta A\|\leq 1/\|A^{-1}\|$ sem
tryggir að fylkið $\tilde A$ sé andhverfanlegt.

\pause
\smallskip
Nú stillum við upp jöfnuhneppinu 
$\tilde A\mathbf{\tilde x}=\mathbf{\tilde  b}$ á forminu
$$
(A+\delta A)(\xv+\delta \xv)=(\bv+\delta\bv)
$$
sem jafngildir
$$
\delta\xv=A^{-1}\big(\delta\bv-(\delta A)\xv-(\delta A)(\delta x)\big)
$$
\end{frame}

\begin{frame}{3.4 Áhrif gagnaskekkju} 

Við vorum komin með jöfnuna
$$
\delta\xv=A^{-1}\big(\delta\bv-(\delta A)\xv-(\delta A)(\delta x)\big).
$$

\pause
\smallskip
Af henni leiðir ójafnan
$$
\|\delta\xv\|\leq 
\|A^{-1}\|\big(\|\delta\bv\|+\|\delta A\|\|\xv\|+\|\delta A\|\|\delta \xv\|\big)
$$
Einangrum nú $\|\delta \xv\|$,
$$
\|\delta\xv\|\leq 
\dfrac{\|A^{-1}\|}{1-\|A^{-1}\|\|\delta A\|}\cdot 
\big(\|\delta\bv\|+\|\delta A\|\|\xv\|\big)
$$\pause
deilum með $\|\xv\|$ báðum megin,  margföldum síðan með $\|A\|$ í teljara
og nefnara í hægri hliðinni,
$$
\dfrac{\|\delta\xv\|} {\|\xv\|}\leq 
\dfrac{\|A\|\|A^{-1}\|}{1-\|A^{-1}\|\|\delta A\|}\cdot 
\bigg(\dfrac{\|\delta\bv\|}{\|A\|\|\xv\|}+
\dfrac{\|\delta A\|}{\|A\|}\bigg)
$$
\end{frame}

\begin{frame}{3.4 Áhrif gagnaskekkju} 

Við vorum komin með 
$$
\dfrac{\|\delta\xv\|} {\|\xv\|}\leq 
\dfrac{\|A\|\|A^{-1}\|}{1-\|A^{-1}\|\|\delta A\|}\cdot 
\bigg(\dfrac{\|\delta\bv\|}{\|A\|\|\xv\|}+
\dfrac{\|\delta A\|}{\|A\|}\bigg).
$$

\pause
\smallskip
Samkvæmt skilgreiningu er $\kappa(A)=\|A\|\|A^{-1}\|$ og
við höfum auk þess ójöfnuna $\|\bv\|\leq \|A\|\|\xv\|$, en það gefur 
matið á hlutfallslegu skekkjunni sem við sækjumst eftir
$$
\dfrac{\|\delta\xv\|} {\|\xv\|}\leq 
\dfrac{\kappa(A)}{1-\kappa(A)\cdot(\|\delta A\|/\|A\|)}\cdot 
\bigg(\dfrac{\|\delta\bv\|}{\|\bv\|}+
\dfrac{\|\delta A\|}{\|A\|}\bigg).
$$
\end{frame}

\section*{3.5 $LU$-þáttun}

\begin{frame}{3.5  Nokkrar skilgreiningar um fylki} 
\begin{enumerate}
  \item[(i)] Fylkið $A$ nefnist {\it neðra  þríhyrningsfylki} ef öll stök fyrir
ofan hornalínuna í $A$ eru $0$, þ.e.~$a_{ij}=0$ ef $i<j$.
\pause
  \item[(ii)] Fylkið $A$ nefnist {\it efra þríhyrningsfylki} ef öll
    stökin neðan við hornalínuna eru $0$, þ.e.~$a_{ij}=0$ ef $i>j$. 
\pause
  \item[(iii)] Fylkið $A$ nefnist {\it bandfylki} (e.~striped matrix) ef til er
$\beta \leq n-2$ þannig að $a_{ij}=0$ ef $|i-j|>\beta$. Minnsta talan
$\beta$ sem uppfyllir þetta skilyrði kallst á {\it bandvídd} fylkisins
$A$.\pause
  \item[(iv)] Ef $A$ er bandfylki með bandvíddina $1$, þá nefnist $A$ {\it
  þríhornalínufylki}.\pause
  \item[(v)] Fylkið $A$ er sagt vera {\it samhverft } ef 
$a_{ij}=a_{ji}$ fyrir öll $i$ og $j$.\pause
  \item[(vi)] Fylkið $A$ er sagt vera {\it jákvætt ákvarðað }  
  ef $\xv^TA\xv>0$ gildir fyrir alla vigra $\xv\neq 0$ í $\R^n$.
\end{enumerate}
\end{frame}


\begin{frame}{3.5  Úrlausn á jöfnuhneppi með neðra þríhyrningsfylki} 

Ef $A$ er neðra þríhyrningsfylki, þá er úrlausn
jöfnuhneppisins $A\xv = \bv$ auðveld, því hneppið er þá af gerðinni 
\begin{alignat*}{2}
	&a_{11}x_1&&=b_1,\\
	&a_{21}x_1+a_{22}x_2&&=b_2,\\
	&a_{31}x_1+a_{32}x_2+a_{33}x_3&&=b_3,\\
	&\vdots \qquad \qquad   \vdots \qquad 
	\qquad \vdots& \qquad& \vdots\\
	&a_{n1}x_1+a_{n2}x_2+a_{n3}x_3\cdots+a_{nn}x_n&&=b_n,
\end{alignat*}
\end{frame}


\begin{frame}{3.5  Úrlausn á jöfnuhneppi með neðra þríhyrningsfylki} 
Við getum rakið okkur niður línurnar og leyst úr stærðirnar 
$x_1,\dots,x_n$ hverja á eftir annarri
\begin{align*}
	x_1& = b_1/a_{11},\\
	x_2& = (b_2-a_{21}x_1)/a_{22},\\
	x_3& = (b_3-a_{31}x_1-a_{32}x_2)/a_{33}\\
   	&\vdots \qquad \qquad \vdots \qquad \qquad \vdots\\
	x_n& = (b_n-a_{n1}x_1-a_{n2}x_2-\cdots
		-a_{n,n-1}x_{n-1})/a_{nn}.
\end{align*}
\end{frame}


\begin{frame}{3.5  Talning á aðgerðunum við úrlausnina} 
Nú skulum við telja saman fjölda reikningsaðgerða sem þarf til þess
að framkvæma þessa útreikninga.

\pause
\smallskip
Við lítum á samlagningu og frádrátt sem sömu aðgerðina. 
Við þurfum enga samlagningu til að reikna út $x_1$,
eina til þess að reikna út $x_2$, tvær til þess að reikna $x_3$ 
og þannig áfram upp í $n-1$ samlagningu til þess að reikna út $x_n$. 

\pause
\smallskip
Heildarfjöldinn er því 
\begin{equation*}
	1 + 2 + \cdots + n-1 = \tfrac 12 n(n-1) = 
	\tfrac 12n^2-\tfrac 12 n.
\end{equation*}

\pause
Fjöldi margfaldana er sá sami. 


\pause
Við þurfum hins vegar aðeins eina deilingu til þess að reikna út
hverja af  stærðunum $x_1,\dots,x_n$. 

\pause 
Heildarfjöldi reikniaðgerða við úrlausn á línulegu jöfnuhneppi
$Ax=b$, þar sem $A$ er neðra  þríhyrningsfylki er því 
$$\tfrac 12 n^2 - \tfrac 12 n
+ \tfrac 12 n^2 - \tfrac 12 n +n = n^2.$$ 
\end{frame}


\begin{frame}{3.5 Úrlausn á jöfnuhneppi með efra þríhyrningsfylki} 
Hugsum okkur nú að $A$ sé efra þríhyrningsfylki.  Þá verður
jöfnuhneppið
\begin{align*}
	a_{11}x_1+a_{12}x_2+a_{13}x_3+\cdots+a_{1n}x_n& = b_1,\\
	a_{22}x_2+a_{23}x_3+\cdots+a_{2n}x_n& = b_2,\\
	a_{33}x_3+\cdots+a_{3n}x_n& = b_3,\\
	\vdots\quad& \quad \vdots\\
	a_{nn}x_n& = b_n,
\end{align*}

\end{frame}


\begin{frame}{3.5 Úrlausn á jöfnuhneppi með efra þríhyrningsfylki} 
Við getum rakið okkur upp línurnar og fundið $x_n,x_{n-1},\dots,x_1$
hverja af annarri 
\begin{align*}
	x_n& = b_n/a_{nn},\\
	x_{n-1}& = (b_{n-1}-a_{n-1,n}x_n)/a_{n-1,n-1},\\
	\vdots&\qquad \vdots\qquad \qquad \vdots\\
	x_1& = (b_1-a_{12}x_2-a_{13}x_3-\cdots
	-a_{1,n}x_{n})/a_{11}.
\end{align*}

\pause
Aðgerðafjöldinn er sá sami og í úrlausn neðra
þríhyrningshneppisins.
\end{frame}

\begin{frame}{3.5 Línuaðgerðir} 
 Gerum ráð fyrir að við séum að ryðja $4\times 4$ fylki með Gauss-eyðingu og 
að við séum búin með fyrsta dálkinn. Næsta skref er að nota línu 2 til þess 
að losna við stökin í sætum (3,2) og (4,2).
$$A=
\left[\begin{array}{llll}
1 & \cdot & \cdot & \cdot\\
0 & a & \cdot & \cdot\\
0 & b & \cdot & \cdot\\
0 & c & \cdot & \cdot
\end{array}\right]
$$
Lína 3, $l_3$, verður þá að $l_3 - \frac ba l_2$, \pause og \\
líne 4, $l_4$, verður að $l_4 - \frac ca l_2$.

\pause

Þessar tvær aðgerðir má einnig framkvæma með því að margfalda fylkið að ofan frá vinstri með
fylkinu
$$
M_2 = \left[
\begin{array}{llll}
1 & 0 & 0 & 0\\
0 & 1 & 0 & 0\\
0 & -\frac ba & 1 & 0\\
0 & -\frac ca & 0 & 1
\end{array}
\right]
$$
\end{frame}

\begin{frame}{3.5 Ný sýn á Gauss-eyðingu}
 Það að ryðja fylkið eins og hér á undan, felst því í því að margfalda $A$ frá vinstri með þremur fylkjum
$M_1, M_2$ og $M_3$ sem eru þannig að $M_i$ er einingafylkið nema í sætum $(i+1,i),\ldots,(n,i)$ 
eru tölur sem eru hugsanlega frábrugðnar 0.

\smallskip
\pause

Athugum að Gauss-eyðing skilar fylki $U$ á efra þríhyrningsformi. \pause

\smallskip

Við getum því skrifað
$$M_3 M_2 M_1 A = U$$

\pause\smallskip

Almennt, fyrir $n \times n$ fylki þá getum við skrifað
$$
M_{n-1}\cdots M_2 M_1 A = U,
$$
þar sem $M_i$ eru fylki eins og lýst er hér að ofan.
\end{frame}

\begin{frame}{3.5 Nánar um $M_i$}
 Við sjáum að ef 
$$
M_i = \left[
\begin{array}{ccccccc}
1 &   &   &   &   &   &  \\
  & . &   &   &   &   &  \\
  &   & 1 &   &   &   &  \\
  &   & m_{i+1,i} & 1 &   &   &  \\
  &   & . &   & . &   &  \\
  &   & . &   &   & . &  \\
  &   & m_{n,i} &   &   &   & 1
\end{array}\right],
$$
þá er 
$$
M_i^{-1} = \left[
\begin{array}{ccccccc}
1 &   &   &   &   &   &  \\
  & . &   &   &   &   &  \\
  &   & 1 &   &   &   &  \\
  &   & -m_{i+1,i} & 1 &   &   &  \\
  &   & . &   & . &   &  \\
  &   & . &   &   & . &  \\
  &   & -m_{n,i} &   &   &   & 1
\end{array}\right].
$$

\end{frame}


\begin{frame}{3.5 Nánar um $M_i$}
Eins þá er auðvelt að sjá að 
$$
M_i^{-1} M_j^{-1} = \left[
\begin{array}{cccccccc}
1 &   &   &   &   &   &   &  \\
  & . &   &   &   &   &   &  \\
  &   & 1 &   &   &   &   &  \\
  &   & -m_{i+1,i} & . &   &   &   &  \\
  &   & . &   & 1 &   &   &  \\
  &   & . &   & -m_{j+1,j} & . &   &  \\
  &   & . &   & . &   & . &  \\
  &   & -m_{n,i} &   & -m_{n,j} &   &   & 1
\end{array}
\right]
$$\pause
Það er
$$
M_1^{-1} M_2^{-1} \cdots M_{n-1}^{-1} = \left[
\begin{array}{cccccc}
1 &   &   &   &   &     \\
-m_{2,1} & 1 &   &   &     &  \\
-m_{3,1} & -m_{3,2} & 1 &    &   &  \\
  & -m_{4,2} & -m_{4,3} & 1   &   &  \\
  & . &   &   & . &    \\
-m_{n,1} & -m_{n,2} & -m_{n,3} & . & -m_{n,n-1}  & 1
   \end{array}
\right]
$$

\end{frame}


\begin{frame}{3.5 $LU$-þáttun}
 Þetta hefur í för með sér að ef við skilgreinum 
$L = M_1^{-1} M_2^{-1} \cdots M_{n-1}^{-1}$ \pause þá er
$$
A = LU
$$\pause
eða
$$
A = \left[\begin{array}{cccccccc}
1 &   &   &   &   &   &   &  \\
-m_{2,1} & 1 &   &   &   &   &   &  \\
-m_{3,1} & -m_{3,2} & 1 &   &   &   &   &  \\
  & -m_{4,2} & -m_{4,3} & 1 &   &   &   &  \\
  & . &   &   & . &   &   &  \\
  & . &   &   &   & . &   &  \\
  & . &   &   &   &   & . &  \\
-m_{n,1} & -m_{n,2} & -m_{n,3} & . & . & . & -m_{n,n-1} & 1
   \end{array}\right] U
$$
Þannig að með því að framkvæma Gauss-eyðingu á $A$ og halda utanum um aðgerðirnar ($M_i$'in)
og niðurstöðuna $U$ þá fæst $LU$-þáttun á $A$.
\end{frame}

\begin{frame}{3.5 $LU$-þáttun og sköluð hlutvending}
\begin{block}{Vandamálið}
  Aðferðin hér að framan gerði ráð fyrir að stak $a_{i,i}$ yrði aldrei 0 (þá
getum við ekki notað þá línu til þess að eyða). Eins hugsuðum við ekkert út í styttingarskekkjur
sem við búum til. 
\end{block}

\begin{block}{Lausnin} 
 Ef við framkvæmum Gauss-eyðinguna með skalaðri hlutvendingu þá ráðum við bót á báðum þessum
atriðum, því þá veljum við aldrei línu með forystustuðul 0 og við minnkum skekkjuna eins og hefur
komið fram áður.
\end{block}

\begin{block}{Athugasemd}
 Þegar við notum skalaða hlutvendingu þá uppfylla fylkin $L$ og $U$ ekki endilega $LU=A$ 
(sjá dæmi bls.~196). \pause Þess í stað fæst 
  $$LU =PA$$
þar sem fylkið $P$ umraðar línunum í $A$ í samræmi við umröðunarvigrinn $\rv$. \pause
Það er, stökin í $P$ eru 0, nema $p_{i,r_i} = 1$.
\end{block}
\end{frame}


\begin{frame}{3.5 Úrlausn $A\xv=\bv$:} 

Við skiptum nú úrlausnarferlinu á $A\xv=\bv$ í þrjú skref

(i) {\bf $LU$-þáttun:}  Reiknum út neðra þríhyrningsfylki $L$  og efra þríhyrningsfylki $U$ með 
skalaðri hlutvendingu. Höldum utanum $\rv$ (og þar með $P$). Þá er 
$$
LU=PA.
$$\pause

(ii) {\bf Forinnsetning:}  Leysum $L\yv=P\bv$.

(iii) {\bf Endurinnsetning:} Leysum $U\xv=\yv$.

Lausnin sem við leitum að er þá $\xv$, því
$$
P\bv = L\yv = UL\xv = PA\xv,
$$
sem er jafngilt því að $\bv=A\xv$
\end{frame}


\begin{frame}{3.5 Fjöldi reikniaðgerða fyrir $LU$-þáttun}
Heildarfjöldi reikningsaðgerða til þess að framkvæma
$LU$-þáttunina er
$$
	\frac 23n^3-\frac 12n^2-\frac 76 n.
$$
Liðir (ii) og (iii) krefjast svo $n^2 + n^2 = 2n^2$ aðgerða til viðbótar.
Samanlagður fjöldi aðgerða er því 
$$
	\frac 23n^3+\frac 32n^2-\frac 76 n.
$$
\pause
\smallskip
Ef $n$ er stór tala, segjum $n=1000$, þá er fyrsti liðurinn lang
stærstur og við getum slegið á aðgerðafjöldann með $\tfrac 23n^3$. 
\pause

Þetta er töluvert betra heldur en að reikna $A^{-1}$ og svo $\xv=A^{-1}\bv$,
en þá er heildafjöldi aðgerða  $2n^3$.
\end{frame}

\begin{frame}{3.5 Mörg jöfnuhneppi}
 Ef við þurfum að leysa mörg jöfnuhneppi með sama stuðlafylkið þá koma
kostir $LU$-þáttunar vel í ljós.
\pause
\smallskip

Gefið $A$ og $\bv_1,\ldots,\bv_m$ þá leitum við að vigrum $\xv_1,\ldots,\xv_m$ þannig að 
$$
A\xv_i = \bv_i, \qquad \text{fyrir } i = 1,\ldots,m.
$$

\pause
\smallskip

Við þurfum bara að framkvæma $LU$-þáttunina einu sinni, en innsetningarnar í 
lið (ii) og (iii) framkvæmum við $m$-sinnum.
Heildar fjöldi aðgerða er þá 
$$
	\frac 23n^3+(2m-\frac 12)n^2- (m-\frac 16) n.
$$
\end{frame}


\section*{3.8 Fastapunktsaðferðir fyrir línuleg jöfnuhneppi}

\begin{frame}{3.8 Ítrekunaraðferðir til þess að leysa línuleg jöfnuhneppi}  

Munum að samanlagður fjöldi reikniaðgerða 
sem þarf til þess að leysa $n\times n$ línulegt jöfnuhneppi
$A\xv=\bv$  með Gauss-eyðingu, for-  og endurinnsetningu er
$\sim \tfrac 23 n^3$.

\pause
\smallskip
Ef jöfnuhneppið er jafngilt hneppinu
$$
\xv=T\xv+\cv
$$
þá getum við sett upp fastapunktsferð til þess að leysa þetta hneppi
með því að giska á eitthvert nálgunargildi $\xv^{(0)}$ fyrir lausnina 
og ítra síðan með formúlunni 
$$
\xv^{(k+1)}=T\xv^{(k)}+\cv, \qquad k=0,1,2,\dots,
$$
í þeirri von að runan $(\xv^{(k)})$ stefni á réttu lausnina $\xv$ á
upprunalega jöfnuhneppinu.

\pause
\smallskip
Það þarf $n^2-n$ aðgerðir til þess að reikna út margfeldið 
$T\vv$ fyrir vigur $\vv\in \R^n$ og því getum við komist upp með að
taka $\approx \tfrac 23 n$ ítrekanir áður en heildaraðgerðafjöldinn
er kominn upp fyrir aðgerðafjöldann í Gauss-eyðingu, ásamt
for- og endurinnsetningu.
\end{frame}

\begin{frame}{3.8 Fastapunktsítrekun til þess að leysa línuleg jöfnuhneppi}  

Við ætlum nú að gera ráð fyrir að jafnan $A\xv=\bv$ sé jafngild
$$
\xv=T\xv+\cv
$$
giskum á eitthvert nálgunargildi $\xv^{(0)}$ fyrir lausnina $\xv$ og
skilgreinum síðan rununa 
$$
\xv^{(k+1)}=T\xv^{(k)}+\cv, \qquad k=0,1,2,\dots,
$$

\pause
\smallskip
Allt er nú undir því komið  að $n\times n$ fylkið $T$ sé vel valið.
\end{frame}

\begin{frame}{3.8 Fastapunktsítrekun - skekkjumat} 

Við skilgreinum nú skekkjuna í $k$-ta ítrekunarskrefinu
$\ev^{(k)}=\xv-\xv^{(k)}$.  Þá gildir formúlan
$$
\ev^{(k)}=T\ev^{(k-1)}=T^2\ev^{(k-2)}=\cdots=T^{k}\ev^{(0)}
$$
sem við höfum áður séð í athugun okkar á fastapunktsaðferðinni.

\pause
\smallskip
Nú beitum við 
$$
\|\ev^{(k)}\| \leq \|T^{k}\|\|\ev^{(0)}\|\leq \|T\|^{k}\|\ev^{(0)}\|
$$

\pause
\smallskip
Við höfum $\xv^{(1)}-\xv^{(0)}=T\xv^{(0)}+\cv-\xv^{(0)}$ og 
$\cv=\xv-T\xv$ og þar með
$$
\xv^{(1)}-\xv^{(0)}=T(\xv^{(0)}-\xv)-(\xv^{(0)}-\xv)=
-(\ev^{(0)}-T\ev^{(0)})=-(I-T)\ev^{(0)}.
$$

\pause
\smallskip
Þetta gefur jöfnuna:
$$
\ev^{(0)}=-(I-T)^{-1} \big(\xv^{(1)}-\xv^{(0)}\big).
$$

\end{frame}

\begin{frame}{3.8 Fastapunktsítrekun - skekkjumat} 

Með smá útreikningi má sýna fram á að ef $\|T\|<1$, þá er
$$
\|(I-T)^{-1}\|\leq \dfrac 1{1-\|T\|}.
$$

\pause
\smallskip
Við vorum komin með ójöfnurnar
$$
\|\ev^{(k)}\| \leq \|T\|^{k}\|\ev^{(0)}\|
$$
og niðurstaðan verður því 
$$
\|\ev^{(k)}\| \leq \dfrac{\|T\|^{k}}{1-\|T\|} 
\|\xv^{(1)}-\xv^{(0)}\|
$$

\pause
\smallskip

Sem þýðir að fastapunktsaðferðin er samleitin þegar $\|T\|<1$.

\end{frame}

\begin{frame}{3.8 Skilyrði fyrir samleitni} 

Munum nú að $\rho(T)$ er rófgeisli fylkisins $T$ sem er samkvæmt
skilgreiningu tölugildi á stærsta eigingildi fylkisins $T$.  


\pause
\smallskip
Rifjum líka upp að fyrir sérhvern náttúrlegan fylkjastaðal $\|\cdot\|$
þá er $\rho(T) \leq \|T\|$,  og að fyrir sérhvert
$\epsilon>0$ gildir að hægt er að finna náttúrlegan fylkjastaðal þannig að
$$
\|T\|\leq \rho(T)+\epsilon.
$$

\pause
\smallskip
Sérstaklega gildir í tilfellinu $\rho(T)<1$ að til er náttúrlegur
fylkjastaðall $\|\cdot\|$ þannig að $\|T\|<1$. 
\pause 

Þetta þýðir að fastapunktsaðferðin er samleitin ef $\rho(T) < 1$.
\end{frame}

\begin{frame}{3.8 Skiptingaraðferð (e.~splitting method)}

Við viljum setja upp fastapunktsaðferð til þess að leysa línulega
jöfnuhneppið $A\xv=\bv$ með því að umrita jöfnuna yfir í jafngilda
línulega jöfnu
$$
\xv=T\xv+\cv.
$$ 

\pause
\smallskip
Gerum ráð fyrir að $A=M-N$ þar sem $M$ er andhverfanlegt fylki.
Þá jafngildir $A\xv=\bv$ jöfnunni $M\xv=N\xv+\bv$ og fastapunktsjafnan
er
$$
\xv=M^{-1}N\xv+M^{-1}\bv,
$$
þar sem $T = M^{-1}N$ og $\cv = M^{-1}\bv$.
\pause

\smallskip
Þessi leið til þess að umrita línulega jöfnuhneppið $A\xv=\bv$ yfir í
jafngilda hneppið $\xv=T\xv+\cv$ nefnist {\it skiptingaraðferð} fyrir
línulega jöfnuhneppið $A\xv=\bv$.  
\end{frame}

\begin{frame}{3.8 Jacobi-aðferð}
Við skrifum $A=D-L-U$, þar sem $D$ er hornalínufylkið með hornalínu
$A$, $L$ er neðra þríhyrningsfylki og $U$ er efra þríhyrningsfylki

\pause
\smallskip
Við tökum $M=D$ og $N=L+U$ og fáum þá $T=D^{-1}(L+U)$ og
$\cv=D^{-1}\bv$. 

\smallskip
Þessi skiptingaraðferð er nefnd {\it Jacobi-aðferð}.


\smallskip
Rakningarformúlan er
$$
\xv^{(k+1)}=D^{-1}(L+U)\xv^{(k)}+D^{-1}\bv.
$$
Ef við skrifum hana hnit fyrir hnit, þá fáum við 
fyrir $i=1,2,\dots,n$, 
$$
x_i^{(k+1)}=\dfrac 1{a_{i,i}}\bigg(
b_i-\sum_{j=1}^{i-1}a_{i,j}x_j^{(k)}-\sum_{j=i+1}^n a_{i,j}x_j^{(k)}\bigg)
$$
\end{frame}

\begin{frame}{3.8 Gauss-Seidel-aðferð} 

Augljós endurbót á Jacobi-aðferðinni er að nota gildið $x_i^{(k+1)}$ fyrir
$i<j$ um leið og það hefur verið reiknað.  

\pause
\smallskip
Við það breytist rakningarformúlan í Jacobi-aðferð í
$$
x_i^{(k+1)}=\dfrac 1{a_{i,i}}\bigg(
b_i-\sum_{j=1}^{i-1}a_{i,j}x_j^{(k+1)}-\sum_{j=i+1}^n a_{i,j}x_j^{(k)}\bigg)
$$
Þetta svarar til þess að við veljum skiptingu á $A$ með 
$M=D-L$ og $N=U$ og þar með að
$$
T=(D-L)^{-1}U \qquad \text{ og } \qquad \cv=(D-L)^{-1} \bv
$$
\end{frame}

\begin{frame}{3.8 SOR-aðferð (e.~successive over-relaxation)} 

Það er hægt að hraða samleitni í Gauss-Seidel-aðferð með því að 
taka vegið meðaltal af gildinu $x_i^{(k+1)}$ sem kemur út úr
Gauss-Seidel reikniritinu og næsta gildi á undan með vægisstuðli sem
við táknum með $\omega$.  

\pause
\smallskip
Formúlan verður
$$
x_i^{(k+1)}=(1-\omega)x_i^{(k)}+\dfrac \omega{a_{i,i}}\bigg(
b_i-\sum_{j=1}^{i-1}a_{i,j}x_j^{(k+1)}-\sum_{j=i+1}^n a_{i,j}x_j^{(k)}\bigg)
$$

\pause
\smallskip
Þetta svarar til þess að  við veljum 
$$
M=\dfrac 1\omega D-L \quad { og } \quad   
N=\bigg(\dfrac 1\omega -1\bigg) D+U
$$
og þar með að
$$
T=\bigg(\dfrac 1\omega D-L\bigg)^{-1}
\bigg(\bigg(\dfrac 1\omega-1\bigg)D+ U\bigg) 
\quad \text{ og } \quad \cv=\bigg(\dfrac 1\omega D-L\bigg)^{-1} \bv
$$
\end{frame}


%
\frame{\frametitle{3.8 Samleitni Gauss-Seidel-aðferðar} 
\begin{block}{Setning}
 \begin{itemize}
  \item 
 Gerum ráð fyrir að  $A$ sé  samhverft
rauntölufylki með öll hornalínustökin jákvæð.  Þá er Gauss-Seidel
aðferðin samleitin ef og aðeins ef $A$ er jákvætt ákvarðað.\pause
\item 
Ef fylkið $A$ er jákvætt ákvarðað, þá er Gauss-Seidel-aðferð
samleitin fyrir sérhvert val á upphafságiskun $\xv^{(0)}$.

 \end{itemize}

\end{block}
}


\section*{3.10 Newton-aðferð fyrir jöfnuhneppi}
%
\frame{\frametitle{3.10 Aðferð Newtons fyrir jöfnuhneppi} 

Látum $f_k : I \to \R$, $k = 1, \ldots, n$, þar sem 
$I$ er svæði í $\R^n$ vera samfelld föll.  \pause
Það getur komið sér vel að geta leyst ólínuleg jöfnuhneppi af gerðinni
\begin{equation*}
	\left\{ \begin{array}{c}
		f_1(x_1, \ldots, x_n) = 0 \\
		f_2(x_1, \ldots, x_n) = 0 \\
		\hdots \\
		f_n(x_1, \ldots, x_n) = 0 \\
	\end{array} \right.
\end{equation*}
\pause
Svo heppilega vill til að aðferð Newtons virkar næstum óbreytt fyrir
slík hneppi. 

}
%
%
\frame{\frametitle{3.10 Jacobi-fylki} 

Skilgreinum $\fv : I \to R^n$ með
\begin{equation*}
	\fv (\xv) = \begin{pmatrix} f_1(\xv), & \ldots &, f_n(\xv)
	\end{pmatrix}
\end{equation*}
og gerum ráð fyrir að allar hlutafleiðurnar $\dfrac{\partial
  f_k}{\partial x_j}$ séu til og séu samfelldar. \pause

Táknum Jacobi-fylki
$\fv$ með $\fv'$, það er 
\begin{equation*}
	\fv'(\xv) = \begin{pmatrix}
		\frac{\partial f_1}{\partial x_1}(\xv)
		& \frac{\partial f_1}{\partial x_2}(\xv)
		& \cdots
		& \frac{\partial f_1}{\partial x_n}(\xv) \\
		\vdots & \vdots & \ddots & \vdots \\
		\frac{\partial f_n}{\partial x_1}(\xv)
		& \frac{\partial f_n}{\partial x_2}(x)
		& \cdots
		& \frac{\partial f_n}{\partial x_n}(\xv)
	\end{pmatrix}
\end{equation*}

}
%


%
\frame{\frametitle{3.10 Aðferð Newtons fyrir hneppi} 
Lausn á hneppinu er því vigur $\rv$ þannig að $\fv(\rv) =0$.
\pause

Ef $\rv$ er lausn hneppisins og $\xv_0 \in \R^n$ er upphafságiskun á
$\rv$ má sjá að runan $(\xv_n)$, þar sem 
\begin{equation*}
	\xv_{n+1} = \xv_n + \hv_n^T
\end{equation*}
og $\hv_n^T$ er lausn á jöfnuhneppinu
\begin{equation*}
	\fv'(\xv_n) \hv_n^T = -\fv(\xv_n)
\end{equation*}
stefnir á lausnina $\rv$. \pause

Við getum metið skekkjuna með
\begin{equation*}
	\ev_{n+1} \approx \| \xv_{n+1} - \xv_n \|
\end{equation*}
\pause og forritið okkar helst næstum óbreytt, 
við þurfum aðeins að skipta {\tt abs} út fyrir skipunina 
{\tt norm}.
}
%



%
\begin{frame}[fragile]{3.10 Matlab-forrit fyrir hneppi} 

%\hline
\begin{verbatim}
function x = newtonNullHneppi(f,df,x0,epsilon)
%
%   x = newtonNullHneppi(f,df,x0,epsilon)
%
% Nálgar núllstöð fallsins f:Rn --> Rn með aðferð Newtons.
% Fallið df er Jacobi-fylki f, x0 er upphafságiskun 
% á núllstöð og epsilon er tilætluð nákvæmni.
% x0 verður að vera dálkvigur og f verður að 
% skila dálkvigrum

x = x0; mis = -df(x)\f(x); 
% Ítrum meðan ástæða er til
while (norm(mis) >= epsilon) 
    x = x + mis; 
    mis = -df(x)\f(x);
end
\end{verbatim}
%\hline
\end{frame}

%

%
\frame{\frametitle{3.10  Sýnidæmi} 

Grafið $y=e^x$ sker lokaða ferilinn sem gefinn er með
jöfnunni $x^4+y^2=1$ í tveimur
punktum.  Notið aðferð Newtons til þess að nálga hnit þeirra með 
5 aukastafa nákvæmni. 

\medskip
{\it Lausn:}  
Það er alveg augljóst að punkturinn ${\mathbf r}=(0,1)$ gefur lausn á
jöfnuhneppinu.
Við látum eins og ekkert sé og giskum á  ${\mathbf x}^{(0)}=(0.5,0.75)$
\begin{center}
\begin{tabular}{|l|l|l|l|}\hline 
  $n$  & $x_1^{(n)} $  & $x_2^{(n)}$ & $\|{\mathbf x}^{(n+1)}-{\mathbf
  x}^{(n)}\|$ \\ \hline
   0 &  0.50000000000000 &  0.75000000000000 & \\\hline 
   1 &  0.17270262414568 &  1.10909912528477 &  0.18447541668198 \\\hline 
   2 &  0.01946538693088 &  1.00638822766059 &  0.02028257413953 \\\hline 
   3 &  0.00020831857772 &  1.00002048613263 &  0.00020930163934 \\\hline 
   4 &  0.00000002190660 &  1.00000000020984 &  0.00000002190761 \\\hline 
   5 &  0.00000000000000 &  1.00000000000000 &  0.00000000000000 \\\hline 
   6 & -0.00000000000000 &  1.00000000000000 &  0.00000000000000 \\\hline 
\end{tabular}
\end{center}

}
%



\frame{\frametitle{3.10  Sýnidæmi framhald} 

Við tökum fyrir hinn skurðpunktinn:
\begin{center}
\begin{tabular}{|l|l|l|l|}
\hline 
  $n$  & $x_1^{(n)} $  & $x_2^{(n)}$ & $\|{\mathbf x}^{(n+1)}-{\mathbf
  x}^{(n)}\|$  \\ \hline 
   0&  -0.80000000000000 &  0.25000000000000 & \\\hline 
   1&  -1.03486380522268 &  0.34379785380788 &  0.07380487278262 \\\hline 
   2&  -0.96968875917544 &  0.37842981331349 &  0.00919771982977 \\\hline 
   3&  -0.96137076039507 &  0.38235523639344 &  0.00014098927991 \\\hline 
   4&  -0.96124395918305 &  0.38241687590740 &  0.00000003252214 \\\hline 
   5&  -0.96124392995056 &  0.38241689016050 &  0.00000000000000 \\\hline 
   6&  -0.96124392995055 &  0.38241689016050 &  0.00000000000000 \\\hline 
\end{tabular}
\end{center}
Nálgun okkar á skurðpunkti ferlanna í vinstra hálfplaninu er
$(-0.96124392995055, 0.38241689016050)$

\medskip
Í töfluna var ekki hægt að koma fyrir athugun á samleitnistiginu
en hlutfallið
$$
\dfrac{\|e_{n+1}\|}{\|e_n\|}
\approx\dfrac{\|{\mathbf x}^{(n+1)}-{\mathbf x}^{(n)}\|}{\|{\mathbf
  x}^{(n)}-{\mathbf x}^{(n-1)}\|^2}
=1.6
$$ 
fyrir fjögur síðustu gildin.
}
%
\section*{3.8 Fastapunktssetningi fyrir jöfnuhneppi}
\frame{\frametitle{3.8  Fastapunktssetning} 

Munum að mengi $X$ í $\R^n$ er sagt vera {\it kúpt} ef strikið sem 
tengir sérhverja tvo punkta í $X$ liggur alltaf í $X$.

\smallskip
{\bf Fastapunktssetning:}

Látum $X$ vera lokað og takmarkað kúpt hlutmengi í $\R^n$ og
$\fv :  X \to X$ vera herpingu, þ.e.a.s.~ til er $\lambda\in [0,1[$
þannig að
$$
\|\fv(\xv)-\fv(\yv)\|\leq \lambda \|\xv-\yv\|, \qquad
\xv,\yv\in X.
$$ Þá hefur $\fv$ nákvæmlega
einn fastapunkt $\rv$ í menginu $X$ og runan $(\xv_n)$ þar sem 
\begin{align*}
  \xv_0 &\in X, \\
  \xv_{n+1} &= \fv(\xv_n), \quad n \geq 0
\end{align*}
stefnir á hann.
}

%

%
\frame{\frametitle{Kafli 3: Fræðilegar spurningar:}

\begin{enumerate}
  \item  Lýsið því hvernig línulegt jöfnuhneppi er leyst með
    $LU$-þáttun, for- og endurinnsetningu.
  \item Hvað þýðir að $A$ sé efra þríhyrningsfylki og hvað þýðir að
    $A$ sé neðra þríhyrningsfylki?
  \item Hvað er bandfylki og hvað er þríhornalínufylki?
  \item Hvað þýðir að $A$ sé samhverft og hvað þýðir að $A$ sé jákvætt
    ákvarðað? 
  \item Hver er heildarfjöldi reikniaðgerða sem þarf til þess
    að leysa $n\times n$ línulegt jöfnuhneppi $A\xv=\bv$ ef $A$ er
    efra eða neðra þríhyrningsfylki? 
  \item Hvað er $LU$-þáttun á $n\times n$ fylki $A$? 
  \item  Hver er stærðargráðan $\approx an^k$ á fjölda langra
    reikningsaðgerða sem þarf til þess að  framkvæma $LU$-þáttun á
    $n\times n$ fylki?
\end{enumerate}
}

%
\frame{\frametitle{Kafli 3: Fræðilegar spurningar:}

\begin{enumerate}
  \item[8.] Hvað er $PLU$-þáttun á fylki $A$ og til hvers er henni beitt?
  \item[9.] Hvað er fylkjastaðall og hvernig er fylkjastaðall sem staðall
$\|\cdot\|_v$ á $\R^ n$ gefur af sér? (Þetta er einnig nefnt
náttúrlegur fylkjastaðall.)
  \item[10.] Hvað er rófgeisli fylkis og hvernig tengist hann fylkjastöðlum?
  \item[11.] Hvernig er ástandstala fylkis skilgreind og hvernig er
    hún notuð til þess að meta hlutfallslega skekkju í nálgunarlausn á
    línulegu jöfnuhneppi $A\xv=\bv$?
  \item[12.] Hvernig er skiptingaraðferð til þess að finna nálgun á
    línulegu jöfnuhneppi?
  \item[13.] Jacobi-aðferð er dæmi um skiptingaraðferð.  Hvernig er
    hún?
  \item[14.] Gauss-Seidel-aðferð er annað dæmi um skiptingaraðferð.
    Hvernig er hún?  
  \item[15.] Hvernig er ítrekunarskrefið í aðferð Newtons fyrir hneppi?
\end{enumerate}
}


\end{document}