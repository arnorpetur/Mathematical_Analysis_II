\lecture[4]{Kafli 4: Eigingildisverkefni}{lecture-text}
\date{24.~janúar 2014}

\begin{document}

\begin{frame}
	\maketitle
\end{frame}

\section*{}
\begin{frame}{Yfirlit}
\begin{block}{Kafli 4: Eigingildisverkefni}
\begin{center}
\begin{tabular}{|l|l|l|l|}\hline
Nr. &Viðfangsefni & Bls. & Glærur\\
\hline
4.0 & Eigingildi og eiginvigrar & 261-264 & 3-5\\
4.1 & Veldaaðferð & 265-280 & 6-12\\
4.2 & Öfug veldaaðferð & 281-295 & 13-17\\
\hline
\end{tabular}
\end{center}

\end{block}
\end{frame}

\section*{4.0 Eigingildi og eiginvigrar}

\begin{frame}{4. Nálgun á eigingildum og eiginvigrum} 
\begin{block}{Skilgreining}
Látum $A$ vera $n\times n$ fylki.  Munum að  $\lambda\in \C$ nefnist
{\it eigingildi} fylkisins $A$ ef til er $\vv\in \C^n\setminus\{\ov\}$ 
þannig að 
$$
A\vv=\lambda\vv.
$$
\pause
Vigurinn $\vv$ nefnist þá {\it eiginvigur } fylkisins $A$ og við
segjum að hann svari til eigingildisins $\lambda$.
\end{block}

\pause
\begin{block}{Athugasemd}
Eigingildi fylkisins $A$ eru nákvæmlega núllstöðvar kennimargliðunnar
$$
p_A(z)=\det(zI-A), \qquad z\in \C.
$$
\end{block}
\pause

\begin{block}{Athugasemd}
Ef $\vv$ er eiginvigur fylkisins $A$, þá er $\alpha \vv$ einnig
eiginvigur fyrir sérhvert $\alpha\in \C\setminus \{\ov\}$. 
\end{block}
\end{frame}

\begin{frame}{4.0 Gróf staðsetning á eigingildum} 
\begin{block}{Skífusetning Gerschgorins}  
Skilgreinum
$$
r_i=\sum\limits_{{\substack{j=1 \\ j\neq i}}}^n|a_{ij}|,
$$
sem er summan af tölugildum stakanna í línu $i$ {\it utan  hornalínunnar} \pause
og látum
$$
C_i=\{z\in \C\,;\, |z-a_{ii}|\leq r_i\}
$$ 
tákna skífuna með miðju í $a_{ii}$ og geislann $r_i$. \pause 
Þá gildir
\begin{enumerate}
\item[(i)] Öll eigingildi $A$ liggja í sammengi skífanna $C_i$.
\pause
\item[(ii)] Ef $k$ af skífunum $C_i$ mynda samanhangandi svæði $R$ 
í $\C$ sem er sundlægt við hinar $n-k$ skífurnar, þá inniheldur $R$ 
nákvæmlega $k$ eigingildi.
\end{enumerate}
\end{block}
\end{frame}

\begin{frame}{4.0 Eiginvigragrunnar} 
Nokkrar staðreyndir um eigingildi og eiginvigra:\pause
\begin{enumerate}
\item[(i)]  Eiginvigrar sem svara til ólíkra eigingilda eru línulega
  óháðir. 
\pause
\item[(ii)] Eiginvigrar sem svara til eins ákveðins eigingildis 
$\lambda$ spanna hlutrúm í $\C^n$.
\pause
\item[(iii)] Við segjum að fylkið $A$ sé \emph{hornalínugeranlegt} ef
  til eru eigingildi $\lambda_1,\lambda_2,\dots,\lambda_n$ og
  tilsvarandi eiginvigrar $\vv_1,\vv_2,\dots,\vv_n$ sem mynda grunn í
  $\R^n$.\pause Þá er hægt að skrifa 
$$
A=T\Lambda T^{-1}
$$
þar sem $\Lambda$ er hornalínufylki með eigingildin
$\lambda_1,\dots,\lambda_n$ á hornalínunni og $T$ er $n\times n$ fylki
þannig að dálkur nr.~$k$ í því samanstendur af hnitum $\vv_k$ miðað
við staðalgrunninn í $\R^n$.
\pause
\item[(iv)] Ef fylkið $A$ er samhverft, þá er það hornalínugeranlegt.
\end{enumerate}
\end{frame}

\begin{frame}{4.1 Veldaaðferð} 
Hugsum okkur nú að við $A$ sé hornalínugeranlegt og að við röðum 
eigingildunum á hornalínu $\Lambda$ í minnkandi röð eftir tölugildi
$$
|\lambda_1|\geq |\lambda_2|\geq \cdots\geq |\lambda_n|
$$
\pause
Tökum einhvern vigur $\xv^{(0)}$ og lítum á liðun hans í eiginvigra
$$
\xv^{(0)}=\alpha_1\vv_1+\cdots+\alpha_n\vv_n
$$
\pause
Skilgreinum síðan rununa $\big(\xv^{(m)}\big)$ með ítruninni
$$
\xv^{(m+1)}=A\xv^{(m)}.
$$
\end{frame}

\begin{frame}{4.1 Veldaaðferð} 
Við fáum þá 
\begin{align*}
\xv^{(1)} =A\xv^{(0)}&=\alpha_1A\vv_1+\cdots+\alpha_nA\vv_n\\
&=\alpha_1\lambda_1\vv_1+\cdots+\alpha_n\lambda_n\vv_n,
\end{align*}
\pause
\begin{align*}
\xv^{(2)}=A\xv^{(1)}&=\alpha_1\lambda_1A\vv_1+\cdots+\alpha_n\lambda_nA\vv_n,\\
&=\alpha_1\lambda_1^2\vv_1+\cdots+\alpha_n\lambda_n^2\vv_n\\ 
& \qquad  \vdots\qquad  \vdots \qquad \vdots\\
\xv^{(m)}&=\alpha_1\lambda_1^m\vv_1+\cdots+\alpha_n\lambda_n^m\vv_n
\end{align*}
\pause
Síðasti vigurinn er 
$$
\xv^{(m)}=  \lambda_1^m 
\big(\alpha_1\vv_1+(\lambda_2/\lambda_1)^m\alpha_2\vv_+\cdots+
(\lambda_n/\lambda_1)^m \alpha_n\vv_n\big)
$$
\end{frame}


\begin{frame}{4.1 Veldaaðferð} 
Við vorum komin með 
$$
\xv^{(m)}=  \lambda_1^m 
\big(\alpha_1\vv_1+(\lambda_2/\lambda_1)^m\alpha_2\vv_2+\cdots+
(\lambda_n/\lambda_1)^m \alpha_n\vv_n\big)
$$
\pause
Hnit númer $i$ í þessum vigri er: 
$$
x_i^{(m)}=  \lambda_1^m 
\big(\alpha_1v_{1,i}+(\lambda_2/\lambda_1)^m\alpha_2v_{2,i}+\cdots+
(\lambda_n/\lambda_1)^m \alpha_nv_{n,i}\big)
$$
\pause
Hugsum okkur nú að $|\lambda_1|>|\lambda_2|$. Þá fæst:
$$
\dfrac{x_i^{(m)}}{x_i^{(m-1)}}
=
\dfrac{\lambda_1^m\big(\alpha_1v_{1,i}+O((\lambda_2/\lambda_1)^m)\big)}
{\lambda_1^{m-1}\big(\alpha_1v_{1,i}+O((\lambda_2/\lambda_1)^{m-1})\big)}
$$
\pause
Ef við höfum $\alpha_1v_{1,i}\neq 0$, þá er niðurstaðan
$$
\dfrac{x_i^{(m)}}{x_i^{(m-1)}}
=\lambda_1
\dfrac{\big(1+O((\lambda_2/\lambda_1)^m)\big)}
{\big(1+O((\lambda_2/\lambda_1))^{m-1}\big)} \to \lambda_1
\quad \text{ þegar }  \quad m\to \infty.
$$\end{frame}

\begin{frame}{4.1 Veldaaðferð} 
Skoðum aftur
$$
\xv^{(m)}=  \lambda_1^m 
\big(\alpha_1\vv_1+(\lambda_2/\lambda_1)^m\alpha_2\vv_2+\cdots+
(\lambda_n/\lambda_1)^m \alpha_n\vv_n\big)
$$
\pause
Ef $|\lambda_1|>|\lambda_2|$, þá gildir fyrir $j > 1$ að  
$(\lambda_j/\lambda_1)^m \to 0$ þegar $m \to \infty$ og \pause
$$
\lim_{m\to \infty} \frac{\xv^{(m)}}{\lambda_1^m} = \alpha_1 \vv_1.
$$
Þannig að ef $\xv^{(0)}$ var valinn í upphafi þannig að $\alpha_1 \neq 0$,
þá skilar þetta eiginvigrinum $\alpha_1\vv_1$ fyrir eigingildið $\lambda_1$.

\end{frame}



\begin{frame}{4.1 Reiknirit til þess að ákvarða stærsta eigingildi fylkis} 
Þegar við reiknum $\xv^{m}$ eins og hér að framan þá er ekki ólíklegt að 
við lendum í undir- eða yfirflæðisvillum ef lengd $\xv$ (skv.~einhverjum staðli)
stefnir á 0 eða $+\infty$. Til þess að ráða bót á þessu þá stöðlum við vigurinn
í hverju skrefi á eftirfarandi hátt.

\pause

\smallskip
Við veljum $\xv^{(0)}$ með einhverjum hætti og skilgreinum síðan
$$
\yv^{(m)}=A\xv^{(m-1)}, \qquad \pause \text{ og svo }
\xv^{(m)}=\dfrac{\yv^{(m)}}{y_{p_m}^{(m)}} \qquad 
$$
\pause
þar sem $p_m$ er númerið á því hniti í $\yv^{(m)}$ sem hefur stærst
tölugildi, sem þýðir að það hnit $p_m$ uppfyllir
$$
|y_{p_m}^{(m)}|=\|\yv^{(m)}\|_\infty=\max_{1\leq j\leq n}|y_j^{(m)}|.
$$
\pause
Ef mörg númer uppfylla þetta skilyrði, þá tökum við bara $p_m$ sem
lægsta gildið á $j$ þar sem jafnaðarmerki gildir \pause
(enda skiptir það ekki máli fyrir skilgreininguna á $\xv^{(m)}$).
\end{frame}

\begin{frame}{4.1 Samleitnin} 
Nú kemur í ljós að 
$y_{p_{m-1}}^{(m)}$ stefnir á $\lambda_1$. 
Auk þess stefnir $\xv^{(m)}$ á eiginvigur sem svarar til $\lambda_1$ og hefur 
lengdina $1$ í $l_\infty$ staðlinum.

\pause
\smallskip
Í útreikningum skilgreinum við því rununa
$\lambda^{(m)}=y_{p_{m-1}}^{(m)}$.
Við gefum okkur síðan þolmörk á skekkju $TOL$ og reiknum úr runurnar
þar til eitt af stoppskilyrðunum gildir: 
\begin{align*}
|\lambda^{(m)}-\lambda^{(m-1)}|&<TOL \qquad \text{ eða } \\
\|\xv^{(m)}-\xv^{(m-1)}\|&<TOL \qquad \text { eða } \\
\|A\xv^{(m)}-\lambda^{(m)}\xv^{(m)}\|&<TOL.
\end{align*}
\end{frame}

\begin{frame}{4.1 Samhverf fylki} 
Munum að ef $A$ er samhverft, þá hefur $A$ eiginvigragrunn og
eiginvigrar sem svara til ólíkra eigingilda eru hornréttir. 

\pause
\smallskip
Í þessu tilfelli er einfaldara að smíða reiknirit svona:
\begin{align*}
  \yv^{(m)}&=A\xv^{(m-1)}\\
\lambda^{(m)}&={\xv^{(m-1)}}^T\yv^{(m)}\\
\xv^{(m)}&= \frac{\yv^{(m)}}{\sqrt{(\yv^{(m)})^T\yv^{(m)}}}
\end{align*}

\pause
Samleitnin verður sú sama:  $\lambda^{(m)}$ stefnir á stærsta
eigingildið og $\xv^{(m)}$ stefnir á tilsvarandi eiginvigur.
\end{frame}

\begin{frame}{4.2 Meira um eigingildi og eiginvigra} 
\begin{block}{Setning} 
Látum sem fyrr $A$ vera $n\times n$ fylki, $\lambda_1,\dots,\lambda_n$
vera eigingildi og $\vv_1,\dots,\vv_n$ vera  tilsvarandi eiginvigra.
\begin{enumerate}
\item [(i)] Látum 
$p(x)=a_0+a_1x+\cdots+a_mx^m$ vera margliðu og skilgreinum $n\times n$
fylkið $B$ með því að stinga $A$ inn í $p$,
$$
B=p(A)=a_0I+a_1A+\cdots+a_mA^m
$$
Þá eru tölurnar $p(\lambda_1),\dots,p(\lambda_n)$ eigingildi fylkisins
$B=p(A)$ með tilsvarandi eiginvigrum $\vv_1,\dots,\vv_n$.
%\pause
\item [(ii)] Ef $A$ er andhverfanlegt þá eru
$1/\lambda_1,\dots,1/\lambda_n$ eigingildi $A^{-1}$ 
með tilsvarandi eiginvigrum $\vv_1,\dots,\vv_n$.
\end{enumerate}
\end{block}
\end{frame}

\begin{frame}{4.2 Andhverf veldaaðferð} 
Af síðustu setningu leiðir að fylkið $B=(A-qI)^{-1}$ hefur eigingildin
$$
\mu_1=\dfrac 1{\lambda_1-q},\ 
\mu_2=\dfrac 1{\lambda_2-q},\ \cdots \  
\mu_n=\dfrac 1{\lambda_n-q}.\ 
$$
Hugsum okkur nú að við viljum finna nálgunargildi fyrir 
eigingildið $\lambda_k$ og að við vitum út frá setningu Gerschgorins 
skífunum nokkurn veginn hvar það er staðsett.  

\pause
\smallskip
Ef við erum með $q$
nógu nálægt $\lambda_k$, þá verður $\mu_k$ stærsta eigingildi
fylkisins $B=(A-qI)^{-1}$

\pause
\smallskip
Þá getum við beitt veldaaðferðinni til þess að búa til runu 
$\mu^{(m)}\to \mu_k$ og við fáum að
$$
\lambda^{(m)}=\dfrac 1{\mu^{(m)}}+q\to \lambda_k. 
$$
\end{frame}

\begin{frame}{4.2 Andhverf veldaaðferð} 
Ef veldaaðferðinni er beitt á fylkið $B=(A-qI)^{-1}$ þá þurfum við að 
reikna út $\yv^{(m)}=(A-qI)^{-1}\xv^{(m-1)}$ í hverju skrefi.

\pause
\smallskip
Þetta er gert þannig að fyrst framkvæmum við $LU$-þáttun á 
fylkinu $LU=(A-qI)$ og framkvæmum síðan for- og endurinnsetningu
til þess að leysa $LU\yv^{(m)}=x^{(m-1)}$. 

\pause
\smallskip
Tölulegar aðferðir fyrir LU-þáttun eru í kafla 3, og verður fjallað um síðar.
\end{frame}

\begin{frame}{4.2 Reiknirit til þess að nálga eigingildi og eiginvigra} 
Takmarkið er að finna nálgun á eigingildinu $\lambda_k$.  
  \begin{enumerate}
  \item [(i)] Finnum  $q\in \R$ sem liggur næst eigingildinu
    $\lambda_k$ af öllum eigingildum $A$
\pause
  \item [(ii)] Þáttum $LU=A-qI$.
\pause
  \item [(iii)]Við veljum $\xv^{(0)}$ með einhverjum hætti og 
leysum  síðan $\yv^{(m)}$ út úr jöfnunni
$$
LU\yv^{(m)}=\xv^{(m-1)}.
$$
\pause
  \item [(iv)] Skilgreinum
$
\xv^{(m)}={\yv^{(m)}}/{y_{p_m}^{(m)}}
$
þar sem $p_m$ er númerið á því hniti í $\yv^{(m)}$ sem hefur stærst
tölugildi, sem þýðir að það hnit uppfyllir
$$
|y_{p_m}^{(m)}|=\|\yv^{(m)}\|_\infty=\max_{1\leq j\leq n}|y_j^{(m)}|.
$$
Ef mörg númer uppfylla þetta skilyrði, þá tökum við bara $p_m$ sem
lægsta gildið á $j$ þar sem jafnaðarmerki gildir.
  \end{enumerate}
\end{frame}

\begin{frame}{4.2 Reiknirit til þess að ákvarða eigingildi} 
Niðurstaðan verður að 
$$
\lambda^{(m)}=\dfrac 1{y_{p_{m-1}}^{(m)}}+q \to \lambda_k
$$
og $\xv^{(m)}$ stefnir á tilsvarandi eiginvigur.
\end{frame}

\begin{frame}{Kafli 4: Fræðilegar spurningar}

\begin{enumerate}
    \item Hvernig er setning Gerschgorins um staðsetningu
    eigingilda fylkis?
  \item Hvernig er veldaaðferð til þess að nálga það
    eigingildi fylkis sem hefur stærst tölugildi?
  \item Afhverju skilgreinum $\xv^{(m)} = \frac{\yv^{(m)}}{y_{p_m}^{(m)}}$ þar sem
  $\yv^{(m)} = A \xv^{(m-1)}$, en ekki bara $\xv^{(m)} = A \xv^{(m-1)}$?
  \item Hvernig er andhverf veldaaðferð til þess að nálga
    eigingildi fylkis? 
  \item Hvernig er skynsamlegast að velja $q$ í andhverfu veldaaðferðinni ef við viljum
finna eigingildið $\lambda_k$?
\end{enumerate}
\end{frame}


\end{document}