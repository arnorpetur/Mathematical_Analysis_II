\lecture[6]{Kafli 6: Töluleg diffrun og heildun}{lecture-text}
\date{21., 26.~og 28.~febrúar, 5., og 7.~mars, 2014}

\begin{document}

\begin{frame}
	\maketitle
\end{frame}

\begin{frame}{Yfirlit}
\begin{block}{Kafli 6: Töluleg diffrun og heildun}
\begin{center}
\begin{tabular}{|l|l|l|l|}\hline
  Kafli & Viðfangsefni & Bls. & Glærur\\
  \hline
  6.0-6.1 &Inngangur		& 429-437 & 3-4\\
  6.2 &Töluleg deildun		& 438-445 & 5-19\\
  6.3 &Richardson-útgiskun	& 447-452 & 20-31\\
  6.4-6.5 &Töluleg heildun	& 455-478 & 32-56\\
  6.7 &Romberg-heildun		& 496-502 & 57-64\\
\hline
\end{tabular}
\end{center}
\end{block}
\end{frame}

%%%%%%%%%%%%%%%%%%%%%%%%%%%%%%%%%%%%%%%%%%%%%%%%%%%%%%%%%%%%%%

\section*{6.0 Inngangur}

\begin{frame}{6.0 Töluleg deildun og heildun  }  
Deildun og heildun eru meginaðgerðir stærðfræðigreiningarinnar.

Þess vegna er nauðsynlegt að geta nálgað
\begin{equation*}
  f'(a),f''(a),f'''(a),\dots \quad 
  \text{ og } \quad
  \int_a^b f(x)\, dx,
\end{equation*}
þar sem $f$ er fall sem skilgreint er á bili $I$ sem inniheldur $a$ og $b$.  
\end{frame}


\begin{frame}{6.0 Meginhugmynd í öllum nálgunaraðferðunum}
Látum $p$ vera margliðu sem nálgar $f$, \pause
og látum $r(x)=f(x)-p(x)$ tákna skekkjuna í nálgun á $f(x)$ með $p(x)$.  Þá er 
$$
  f'(x)=p'(x)+r'(x), \quad f''(x)=p''(x)+r''(x), \dots 
$$
og
$$ 
  \int_a^b f(x)\, dx=\int_a^b p(x)\, dx+\int_a^b r(x)\, dx.
$$

\pause
\smallskip
Nú þurfum við að gera tvennt:
\begin{enumerate}
\item[(i)] Finna heppilegar nálgunarmargliður og reikna út
$$
  p'(a), \ p''(a),\dots, \qquad \int_a^b p(x)\, dx 
$$
\item[(ii)]  Meta skekkjurnar
$$
  r'(a), \ r''(a), \dots \int_a^b r(x)\, dx
$$
\end{enumerate}
Byrjum á að leiða út nokkrar  nálgunarformúlur með skekkjumati. 
\end{frame}

\section*{6.2 Töluleg diffrun}
\begin{frame}{6.2 Töluleg deildun} 
Látum $f : I \to \mathbb R$ vera fall á bili $I \subset \mathbb R$ og
$a$ vera punkt í $I$. Afleiða $f$ í punktinum $a$ er skilgreind með 
\begin{equation*}
  f'(a) = \lim\limits_{h \to 0}
  \frac{f(a+h)-f(a)}{h}
\end{equation*}
ef markgildið er til.  Við skrifum því oft
\begin{equation*}
  f'(a) \approx \frac{f(a+h)-f(a)}{h}
\end{equation*}

\pause
Þessi nálgun er kölluð {\em frammismunur} því oftast hugsar maður sér
að $h > 0$ og þá er $a+h$ lítið skref áfram frá $a$. 

\pause
\smallskip
Við þurfum skekkjumat fyrir þessa formúlu ef við eigum að geta notað hana.
\end{frame}


\begin{frame}{6.2 Frammismunur}
Við fáum mat á skekkjuna í nálguninni með að skoða Taylor-margliðu 
$f$ í $a$.  Samkvæmt setningu Taylors er til $\xi$ á milli $a$ og
$a+h$ þannig að  
\begin{equation*}
  f(a+h) = f(a) + f'(a)h + \frac{1}{2} f''(\xi)h^2.
\end{equation*}

\pause
Þá fæst að skekkjan í nálgun á $f'(a)$ með  
$$
  \frac{f(a+h)-f(a)}{h} = f[a,a+h]
$$ 
er 
\begin{equation*}
  e = f'(a) - \frac{f(a+h)-f(a)}{h} = -\frac{1}{2} f''(\xi) h
\end{equation*}
Með öðrum orðum
$$
% \min_{t\in [0,h]} \left|-\frac 12 f''(t)\right|h \leq e \leq 
% \max_{t\in [0,h]} \left|-\frac 12 f''(t)\right|h.
\min_{t\in [0,h]} -\frac 12 f''(t)h \leq e \leq 
\max_{t\in [0,h]} -\frac 12 f''(t)h.
$$

\pause
\smallskip
Við sjáum því að $e=O(h)$ þegar $h \to 0$.
\end{frame}


\begin{frame}{6.2 Bakmismunur} 
Við getum sett $a-h$ í stað $a+h$ í skilgreininguna á afleiðu.
Þá fæst svokallaður {\em bakmismunur}
\begin{equation*}
  f'(a) \approx \frac{f(a)-f(a-h)}{h}
\end{equation*}
og ljóst er að sama skekkjumat gengur fyrir þessa nálgun
og fyrir nálgun með frammismun.
\end{frame}


\begin{frame}{6.2 Miðsettur mismunakvóti} 
Lítum nú á þriðja stigs Taylor nálgun 
\begin{align*}
  f(a+h)&=f(a)+f'(a)h+\tfrac 12 f''(a)h^2+\tfrac 16 f'''(\alpha)h^3,\\
  f(a-h)&=f(a)-f'(a)h+\tfrac 12 f''(a)h^2-\tfrac 16 f'''(\beta)h^3,
\end{align*}
þar sem $\alpha$ er á milli $a$ og $a+h$ og $\beta$ er á milli $a$ og
$a-h$.  

\pause
Tökum nú mismuninn og fáum
$$
f(a+h)-f(a-h)=f'(a)\cdot 2h+\tfrac 16\big(f'''(\alpha)+f'''(\beta)\big)h^3
$$
Ef  $f'''$ er samfellt fall, þá gefur
milligildis\-setningin okkur að til er $\xi$ á milli $\alpha$ og $\beta$
þannig að $f'''(\xi)=\tfrac 12 (f'''(\alpha)+f'''(\beta))$ 
\end{frame}


\begin{frame}{6.2 Miðsettur mismunakvóti} 
Niðurstaðan verður 
$$
  f'(a)=\dfrac{f(a+h)-f(a-h)}{2h}-\tfrac 16f'''(\xi)h^2.
$$
Þannig að skekkjan er 
$$
  e = -\frac 16 f'''(\xi) h^2,
$$
og jafnframt er $e = O(h^2)$ þegar $h\to 0$.
\end{frame}


\begin{frame}{6.2 Miðsettur mismunakvóti fyrir aðra afleiðu} 
Við getum útfært þessa sömu hugmynd til þess að reikna út aðra
afleiðu, en þá byrjum við með fjórða stigs Taylor-nálgun 
\begin{align*}
  f(a+h)&=f(a)+f'(a)h+\tfrac 12 f''(a)h^2+\tfrac 16 f'''(a)h^3
+\tfrac 1{24}f^{(4)}(\alpha)h^4,\\
  f(a-h)&=f(a)-f'(a)h+\tfrac 12 f''(a)h^2-\tfrac 16 f'''(a)h^3
+\tfrac 1{24}f^{(4)}(\beta)h^4,
\end{align*}
þar sem $\alpha$ er á milli $a$ og $a+h$ og $\beta$ er á milli $a$ og
$a-h$.  

\pause
\smallskip
Nú leggjum við saman og fáum
$$
f(a+h)+f(a-h)=2f(a) +f''(a)h^2+\tfrac
1{24}\big(f^{(4)}(\alpha)+f^{(4)}(\beta)\big)h^4.
$$
Nú þurfum við að gefa okkur að $f^{(4)}$ sé samfellt fall, þá gefur
milligildissetningin okkur að til er $\xi$ á milli $\alpha$ og $\beta$
þannig að $f^{(4)}(\xi)=\tfrac 12 (f^{(4)}(\alpha)+f^{(4)}(\beta))$. 
\end{frame}


\begin{frame}{6.2 Miðsettur mismunakvóti fyrir aðra afleiðu} 
Niðustaðan verður 
$$
f''(a)=\dfrac{f(a+h)+f(a-h)-2f(a)}{h^2}-\tfrac 1{12}f^{(4)}(\xi)h^2
$$

\pause
\smallskip
Með Taylor-margliðum  má leiða út fleiri nálgunarformúlur fyrir
afleiður.  

\smallskip
Við ætlum ekki að halda lengra í þessa átt heldur snúa okkur að
almennu aðferðinni.
\end{frame}


\begin{frame}{6.2 Almenn aðferð til að nálga afleiður} 
Ef $x_0,\ldots, x_n$ eru punktar í $I$ (hugsanlega með endurtekningum)
og $p$ er margliðan sem brúar $f$ í þeim, þá er 
\begin{equation*}
  f(x) = p(x) + r(x),
\end{equation*}
þar sem  skekkjuliðurinn $r(x)$ er gefinn með formúlunni
$$
r(x)=f[x_0,\ldots,x_n,x](x-x_0)\cdots(x-x_n)
$$
Ef við tökum $p'(a)$ sem nálgun á $f'(a)$ er skekkjan
\begin{equation*}
r'(a) =  f'(a) - p'(a). 
\end{equation*}
\end{frame}


\begin{frame}{6.2 Skekkjumat} 
Munið að formúlan fyrir afleiðu af margfeldi margra þátta er
\begin{multline*}
  (\varphi_1\varphi_2\varphi_3\cdots\varphi_m)'(a)\\
=\varphi_1'(a)\varphi_2(a)\varphi_3(a)\cdots\varphi_m(a)
+\varphi_1(a)\varphi_2'(a)\varphi_3(a)\cdots\varphi_m(a)
+\cdots\\
\cdots+\varphi_1(a)\varphi_2(a)\cdots \varphi_{m-1}(a)\varphi_m'(a)\\
\end{multline*}

\pause
\smallskip
Horfum nú á skekkjuliðinn $r(x)$.  Hann er svona margfeldi með 
$\varphi_1(x)=f[x_0,\dots,x_n,x]$, $\varphi_2(x)=x-x_0$,
$\varphi_3(x)=x-x_1$ o.s.frv.

\pause
\smallskip
Athugum nú að ef $a$ er einn af gefnu punktunum $x_k$, þá er 
$\varphi_{k+2}(x)=(x-x_k)$ sem gefur $\varphi_{k+2}(x_k)=0$ og
$\varphi_{k+2}'(x_k)=1$.  

\smallskip
Þetta segir okkur að ef við tökum $a=x_k$, þá 
eru allir liðirnir í summunni í hægri hliðinni $0$ nema 
einn, þ.e.~við sitjum eftir með þann sem inniheldur $\phi_{k+2}'$.
\end{frame}


\begin{frame}{6.2 Skekkjumat frh.}
Niðurstaðan verður því að skekkjan í nálgun á $f'(a)$ með $p'(a)$ er
\begin{align*}
  f'(a) - p'(a) &= r'(a)
=f[x_0,\dots,x_n,x_k]
\prod_{\stackrel{j=0}{j \not= k}} (x_k-x_j)\\
&=\dfrac{f^{(n+1)}(\xi)}{(n+1)!}
  \prod_{\stackrel{j=0}{j \not= k}} (a-x_j)
\end{align*}
þar sem $a=x_k$.

\pause
\smallskip
Hér notuðum við skekkjumatið fyrir Newton aðferðina (glæra 5.56) sem segir að 
til er $\xi$ á minnsta bilinu sem inniheldur $x_0,\ldots,x_n,x_k$ sem uppfyllir
$$
  f[x_0,\ldots,x_n,x_k] = \frac{f^{(n+1)}(\xi)}{(n+1)!}.
$$
\end{frame}


\begin{frame}{6.2 Frammismunur} 
Nálgum $f$ með fyrsta stigs
brúunarmargliðunni gegnum punktana $(a,f(a))$ og $(a+h,f(a+h))$
(þ.e.~$x_0 = a$ og $x_1 = a+h$),
$$
  f(x)=f[a]+f[a,a+h](x-a)+f[a,a+h,x](x-a)(x-a-h)
$$ 

\pause
\smallskip
Af þessu leiðir formúlan sem við vorum áður komin með
$$
  f'(a)=f[a,a+h]+f[a,a+h,a](a-a-h)
  =\dfrac{f(a+h)-f(a)}h-\tfrac 12 f''(\xi)h
$$ 
Þar sem $\xi$ er á milli 
$a$ og $a+h$ og uppfyllir að $f[a,a+h,a]=f[a,a,a+h]=\tfrac 12f''(\xi)$.
Hér erum við að notafæra okkur aftur skekkjumatið sem við sönnuðum  í kaflanum
um brúunarmargliður.
\end{frame}


\begin{frame}{6.2 Miðsettur mismunakvóti} 
Tökum þriggja punkta brúunarformúlu með $a-h$, $a+h$ og
$a$.  Þá er
\begin{align*}
  f(x)&=f[a-h]+f[a-h,a+h](x-a+h)\\
  &+f[a-h,a+h,a](x-a+h)(x-a-h)\\
  &+f[a-h,a+h,a,x](x-a+h)(x-a-h)(x-a)
\end{align*}

\pause
\smallskip
Athugum að afleiðan af annars stigs þættinum
$$
  x\mapsto (x-a+h)(x-a-h)=(x-a)^2-h^2
$$
er $0$ í punktinum $a$ og því er
\begin{align*}
  f'(a)&=f[a-h,a+h]+f[a-h,a+h,a,a](-h^2)\\
  &=\dfrac{f(a+h)-f(a-h)}{2h}-\tfrac 16 f'''(\xi)h^2 
\end{align*}
Hér nýttum við okkur að til er $\xi$ á milli 
$a-h$ og $a+h$ þannig að $f[a-h,a+h,a,a]=\tfrac 16 f'''(\xi)$.
\end{frame}


\begin{frame}{6.2 Miðsettur mismunakvóti fyrir aðra afleiðu} 
Áfram heldur leikurinn.  
Nú skulum við leiða aftur út formúluna fyrir nálgun á $f''(a)$ með
miðsettum mismunakvóta 

\pause
\smallskip
Þá tökum við þriggja punkta brúunarformúlu með $a-h$, $a+h$ og
$a$ með $a$ tvöfaldan.  Þá er
\begin{align*}
  f(x)&=f[a-h]+f[a-h,a+h](x-a+h)\\
  &+f[a-h,a+h,a](x-a+h)(x-a-h)\\
  &+f[a-h,a+h,a,a](x-a+h)(x-a-h)(x-a)\\
  &+f[a-h,a+h,a,a,x](x-a+h)(x-a-h)(x-a)^2
\end{align*}
Gætum þess að halda liðnum $(x-a)$.  Þá fáum við
\begin{align*}
  f(x)&=f[a-h]+f[a-h,a+h](x-a+h)\\
  &+f[a-h,a+h,a]\big((x-a)^2-h^2)\big)\\
  &+f[a-h,a+h,a,a]\big((x-a)^3-h^2(x-a))\big)\\
  &+f[a-h,a+h,a,a,x]\big((x-a)^4-h^2(x-a)^2)\big)
\end{align*}
\end{frame}


\begin{frame}{6.2 Miðsettur mismunakvóti fyrir aðra afleiðu, frh.} 
Nú þurfum við að reikna aðra afleiðu í punktinum $a$. 
Athugum að önnur afleiða af annars stigs þættinum
$$
  x\mapsto (x-a+h)(x-a-h)=(x-a)^2-h^2
$$
er fastafallið $2$, önnur afleiða af þriðja stigs liðnum
$$
  x\mapsto (x-a)^3-h^2(x-a)
$$
er $0$ í punktinum $a$ og önnur afleiða af fjórða stigs liðnum
$$
  x\mapsto (x-a)^4-h^2(x-a)^2
$$
er fastafallið $-2h^2$.  
\end{frame}


\begin{frame}{6.2 Miðsettur mismunakvóti fyrir aðra afleiðu, frh.} 
Við höfum því 
$$
  f''(a)=2f[a-h,a+h,a]+f[a-h,a+h,a,a,a](-2h^2)
$$
Nú er til punktur $\xi$ á minnsta bili sem inniheldur 
$a-h$, $a+h$ og $a$ þannig að $f[a-h,a+h,a,a,a]=\tfrac
1{24}f^{(4)}(\xi)$.  

\pause
\smallskip
Við þurfum að reikna út fyrri mismunakvótann
\begin{align*}
  f[a-h,a+h,a]&=f[a-h,a,a+h]=\dfrac{f[a,a+h]-f[a-h,a]}{2h}\\
  &=\dfrac 1{2h}\bigg(\dfrac{f(a+h)-f(a)}h-\dfrac{f(a)-f(a-h)}h\bigg)\\
  &=\dfrac{f(a+h)+f(a-h)-2f(a)}{2h^2}  
\end{align*}
Við höfum því leitt aftur út formúluna
$$
  f''(a)=\dfrac{f(a+h)+f(a-h)-2f(a)}{h^2}-\tfrac
  1{12}f^{(4)}(\xi)h^2
$$
\end{frame}


\section*{6.3 Richardson-útgiskun}
\begin{frame}{6.3 Richardson útgiskun} 

Það ætti að vera ljóst að töluleg deildun er nokkuð óstöðug aðferð því
ef skrefastærðin $h$ er lítil eru tölurnar $f(a+h), f(a), f(a-h)$
nálægt hver annarri og við getum lent í styttingarskekkjum. 

\pause
\smallskip
Því er ekki hægt að búast við að fá alltaf betri nálgun á $f'(a)$ við að
minnka skrefalengdina $h$. 

\pause
\smallskip
Leiðin  er Richardson útgiskun (e.~extrapolation), sem er aðferð til
að bæta nálganir.

\pause
\smallskip
Til eru mjög almennar útgáfur þessarar aðferðar en
við munum aðeins skoða þau sértilfelli sem nýtast okkur mest.
\end{frame}


\begin{frame}{6.3 Útleiðsla á miðsettum mismunakvóta} 
Við skulum byrja á að að leiða aftur út formúluna fyrir miðsettann
mismunakvóta til að fá betri upplýsingar um skekkjuliðinn. Fyrir fall
$f$ sem er nógu oft deildanlegt má beita Taylor til að skrifa 
{\small\begin{align*}
  f(a+h) &= f(a) + f'(a)h %+ \frac{f''(a)}{2!}h^2 
  + \ldots
  + \frac{f^{(2n)}(a)}{(2n!)}h^{2n}
  + \frac{f^{(2n+1)}(a)}{(2n+1)!)}h^{2n+1} + O(h^{2n+2}) \\
  f(a-h) &= f(a) - f'(a)h 
  %+ \frac{f''(a)}{2!}h^2 
  + \ldots
  + \frac{f^{(2n)}(a)}{(2n!)}h^{2n}
  - \frac{f^{(2n+1)}(a)}{(2n+1)!)}h^{2n+1} + O(h^{2n+2})
\end{align*}}

\pause
Ef við drögum seinni jöfnuna  frá þeirri fyrri fæst
{\small\begin{equation*}
  f(a+h)-f(a-h) = 2f'(a)h + 2\frac{f'''(a)}{3!}h^3
  + \ldots + 2\frac{f^{(2n+1)}(a)}{(2n+1)!}h^{2n+1} + O(h^{2n+2})
\end{equation*}}\pause
svo ef við einangrum $f'(a)$ sjáum við að
\begin{equation*}
  f'(a) = R_1(h) 
  + a_2 h^2 + a_4 h^4 + \ldots + a_{2n} h^{2n} + O(h^{2n+1})
\end{equation*}
þar sem
\begin{equation*}
  R_1(h) = \frac{f(a+h)-f(a-h)}{2h}
  \quad \text{og} \quad
  a_k = -\frac{f^{(k+1)}(a)}{(k+1)!},
  \quad k = 2,4,\ldots,2n.
\end{equation*}
\end{frame}


\begin{frame}{6.3 Helmingun á skrefinu} 
Hér er minnsta veldi í skekkjuliðnum $h^2$, svo nálgunin $f'(a)
\approx R_1(h)$ er $O(h^2)$, eins og við höfum reyndar séð
áður. Helmingum nú skrefalengdina $h$, þá fæst 
\begin{equation*}
  f'(a) = R_1(h/2) + a_2 \left(\frac{h}{2}\right)^2
  + a_4 \left(\frac{h}{2}\right)^4 + \ldots
  + a_{2n} \left(\frac{h}{2}\right)^{2n} + O(h^{2n+1}).
\end{equation*}
\end{frame}


\begin{frame}{6.3 Helmingun} 
Nú berum við saman þessi tvö skref:
\begin{align*}
  f'(a) &= R_1(h/2) + \tfrac 14 a_2 h^2
  + a_4 \left(\frac{h}{2}\right)^4 + \ldots
  + a_{2n} \left(\frac{h}{2}\right)^{2n} + O(h^{2n+1}),\\
  f'(a) &= R_1(h) 
  + a_2 h^2 + a_4 h^4 + \ldots + a_{2n} h^{2n} + O(h^{2n+1})\\
\end{align*}

\pause
Margföldum efri jöfnuna með $4$ og drögum þá síðari frá.  Þá stendur
eftir
\begin{align*}
  3f'(a) &= 4 R_1(h/2) - R_1(h) 
  + a_4 \left( \frac{4}{2^4} - 1 \right)h^4 \\
  &+ a_6 \left( \frac{4}{2^6} - 1 \right)h^6
  + \ldots
  + a_{2n} \left( \frac{4}{2^{2n}} - 1 \right)h^{2n}
  + O(h^{2n+1})
\end{align*}
\end{frame}


\begin{frame}{6.3 Fjórða stigs nálgun} 
Nú erum við komin með nýja formúlu:
\begin{equation*}
  f'(a) = R_2(h) + b_4 h^4 + b_6 h^6 + \ldots + b_{2n} h^{2n}
  + O(h^{2n+1})
\end{equation*}
þar sem
\begin{equation*}
  R_2(h) = \frac{4 R_1(h/2) - R_1(h)}{3}
  \quad \text{og} \quad
  b_k = \frac{a_k}{3} \cdot \left(\frac{4}{2^k}-1\right),
  \  k = 4,6,\ldots,2n.
\end{equation*}

\pause
Ef við berum þetta saman við jöfnuna sem við byrjuðum með
\begin{equation*}
  f'(a) = R_1(h) 
  + a_2 h^2 + a_4 h^4 + \ldots + a_{2n} h^{2n} + O(h^{2n+1})
\end{equation*}
þá sjáum við að minnsta veldi í skekkjuliðnum er $h^4$, svo nálgunin $f'(a)
\approx R_2(h)$ uppfyllir 
\begin{equation*}
  f'(a) - R_2(h) = O(h^4)
\end{equation*}
og er því betri nálgun en áður. 

\smallskip
 Þetta ferli heitir {\it Richardson útgiskun}.
\end{frame}


\begin{frame}{6.3 Hægt er að halda áfram útgiskun} 
Næsta takmark er að eyða liðnum $b_4h^4$ úr þessari formúlu með því að
líta á 
\begin{equation*}
  f'(a) = R_2(h/2) + b_4 \left(\frac{h}{2}\right)^4 
  + b_6 \left(\frac{h}{2}\right)^6 + \ldots
  + b_{2n} \left(\frac{h}{2}\right)^{2n} + O(h^{2n+1})
\end{equation*}

\pause
Síðan stillum við þessari jöfnu upp með þeirri síðari
\begin{align*}
  f'(a) &= R_2(h/2) + \tfrac 1{16}b_4 h^4 
  + \tfrac 1{64}b_6 h^6 + \ldots
  + \tfrac 1{2^{2n}}b_{2n} h^{2n} + O(h^{2n+1})\\
  f'(a) &= R_2(h) + b_4 h^4 + b_6 h^6 + \ldots + b_{2n} h^{2n}
  + O(h^{2n+1})
\end{align*}

\pause
Margföldum fyrri jöfnuna með $16$ og drögum þá síðari frá 
\begin{align*}
  15f'(a) &= 16 R_2(h/2) - R_2(h) 
  + b_6 \left( \frac{16}{2^6} - 1 \right) h^6 \\
  &+ b_8 \left( \frac{16}{2^8} - 1 \right) h^8
  + \ldots
  + b_{2n} \left( \frac{16}{2^{2n}} - 1 \right) h^{2n}
  + O(h^{2n+1}).
\end{align*}
\end{frame}


\begin{frame}{6.3 Sjötta stigs skekkja} 
\begin{align*}
  15f'(a) &= 16 R_2(h/2) - R_2(h) 
  + b_6 \left( \frac{16}{2^6} - 1 \right) h^6 \\
  &+ b_8 \left( \frac{16}{2^8} - 1 \right) h^8
  + \ldots
  + b_{2n} \left( \frac{16}{2^{2n}} - 1 \right) h^{2n}
  + O(h^{2n+1}).
\end{align*}

\pause
Því er
\begin{equation*}
  f'(a) = R_3(h) + c_6 h^6 + c_8 h^8 \ldots + c_{2n} h^{2n}
  + O(h^{2n+1})
\end{equation*}
þar sem
\begin{equation*}
  R_3(h) = \frac{16 R_2(h/2) - R_2(h)}{15},
  \quad \text{og} \quad
  c_k = \frac{b_k}{15} \cdot \left( \frac{16}{2^k} - 1 \right),
  \quad k = 6,8,\ldots,2n.
\end{equation*}

\pause
Nýja nálgunin uppfyllir
\begin{equation*}
  f'(a) - R_3(h) = O(h^6)
\end{equation*}
og er því enn betri en áður, en við þurfum líka að reikna út
$R_1(h/4)$ til að reikna $R_2(h/2)$. 
\end{frame}


\begin{frame}{6.3 Almenn rakningarformúla} 
Richardson-útgiskunin heldur áfram og út kemur
\begin{equation*}
  R_{i+1}(h) = \frac{4^i R_i(h/2) - R_i(h)}{4^i-1}
  = R_i(h/2) + \frac{R_i(h/2)-R_i(h)}{4^i-1}
\end{equation*}

fyrir $(i+1)$-tu Richardson útgiskun og $R_{i+1}(h)$ uppfyllir að
\begin{equation*}
  f'(a) - R_{i+1}(h) = O(h^{2i+2}),
\end{equation*}
en á móti kemur að til að reikna út $R_{i+1}(h)$ þurfum við að hafa
reiknað út tölurnar 

$R_1(h)$, $R_1(h/2)$, $\ldots$, $R_1(h/2^i)$ auk \\
$R_2(h)$, $R_2(h/2)$, \ldots, $R_2(h/2^{i-1})$ og svo framvegis að\\
$\qquad \vdots$ \\
$R_i(h)$ og $R_i(h/2)$. 

\pause 
\smallskip
Eins og áður sagði fara styttingarskekkjur á endanum að
segja til sín í útreikningum á $R_1(h)$, svo einhver takmörk eru fyrir
hversu margar Richardson útgiskanir er hægt að framkvæma. 
\end{frame}


\begin{frame}{6.3 Reiknirit} 
Útreikningarnir að ofan eru yfirleitt settir fram í töflu
\begin{equation*}
  \begin{array}{ccccc}
    D(1,1) &   &   &   &   \\
    D(2,1) & D(2,2) &  &  &  \\
    D(3,1) & D(3,2) & D(3,3) & & \\
    \vdots & \vdots & \vdots & \ddots & \\
    D(n,1) & D(n,2) & D(n,3) & \ldots & D(n,n)
  \end{array}
\end{equation*}
þar sem $D(i,j) = R_j(h/2^{i-j})$ og þar með
\begin{equation*}
  D(i,j) = \begin{cases}
    \dfrac{f(a+h/2^{i-1})-f(a-h/2^{i-1})}{2\cdot h/2^{i-1}}, & j = 1 \\
    D(i,j-1) + \dfrac{D(i,j-1)-D(i-1,j-1)}{4^{j-1}-1}, & j > 1
  \end{cases}
\end{equation*}
sem gerir auðvelt að forrita Richardson útgiskun. 
\end{frame}


\begin{frame}{6.3 Skekkjumat}
Finnum nú eftirámat fyrir $D(i,j)$ með stærðunum 
$D(i,j-1)$ og $D(i-1,j-1)$. Hér á eftir er $R_j(h/2)$ í hlutverki $D(i,j-1)$ og 
$R_i(h)$ í hlutverki $D(i-1,j-1)$ \\
($h$ er helmingað þegar við förum niður um eina línu).

\pause
\smallskip
Munum að $R_i(h)$ uppfyllir að
\begin{equation*}
  f'(a) = R_j(h) + Kh^{2j} + O(h^{2j+1})
\end{equation*}
fyrir eitthvert $K$ í $\mathbb R$ og að
\begin{equation*}
  f'(a) = R_j(h/2) + K \left( \frac{h}{2} \right)^{2j}
  + O(h^{2j+1})
\end{equation*}

\pause
\smallskip
Ef við tökum mismun á hægri og vinstri hliðum þessara jafna, þá fáum
við 
\begin{equation*}
  0 = R_j(h) - R_j(h/2) + K \left(1 - \frac{1}{2^{2j}}\right)h^{2j}
  + O(h^{2j+1})
\end{equation*}
og ef við einangrum $K$ fæst
$$
  K = -\frac{4^{j}}{h^{2j}} \cdot \frac{R_j(h)-R_j(h/2)}{4^{j}-1} +
O(h^{2j+1}).
$$
\end{frame}


\begin{frame}{6.3 Útleiðsla á fyrirframmati} 
Þá er skekkjan í nálgun á $f'(a)$ með $R_j(h/2)$ jöfn
\begin{align*}
  e_j(h/2) &= f'(a) - R_j(h/2) \\
  &= K\left(\frac{h}{2}\right)^{2j} + O(h^{2j+1}) \\
  &= -\frac{R_j(h)-R_j(h/2)}{4^{j}-1} + O(h^{2j+1}) \\
  &\approx -\frac{R_j(h)-R_j(h/2)}{4^{j}-1}.
\end{align*}

\pause
Þar sem $R_j(h/2)$ er nálgun á $f'(a)$ af stigi $O(h^{2j+1})$, en
$R_{j+1}(h)$ er nálgun á $f'(a)$ af stigi $O(h^{2i+3})$ getum við
slegið á $e_{j+1}(h)$ með $e_j(h/2)$. Ef við lækkum vísinn $j+1$ um
einn gefur það okkur matið 
\begin{equation*}
  e_j(h) \approx \frac{R_{j-1}(h)-R_{j-1}(h/2)}{4^{j-1}-1} =
  \frac{D(i,j-1)-D(i-1,j-1)}{4^{j-1}-1}
\end{equation*}
sem er einmitt liðurinn í rakningarformúlunni fyrir $D(i,j)$. 
\end{frame}


% \begin{frame}{6.3 Lýsing á forriti} 
% Við skulum smíða forrit sem nálgar afleiðu gefins falls með Richardson
% útgiskun. 
% 
% \pause
% Til að geta forritað aðferðina þurfum við einhverja
% upphafsskrefalengd $h_0$ og stöðvunarskilyrði. 
% 
% \pause
% Það er best að hafa tvö  skilyrði; 
% annars vegar stoppum við ef skekkjan fer niður fyrir gefin
% mörk $\varepsilon$ og hins vegar ef fjöldi ítrana fer yfir gefið
% hámark imax. 
% \end{frame}
% 

% \begin{frame}[fragile]{6.3 Matlab-forrit fyrir Ricardson-útgiskun:} 
% \begin{verbatim}
% function [R,mat] = richardson(f,a,h0,epsilon,imax) 
% % 
% %   [R,mat] = richardson(f,a,h0,epsilon,imax)
% % 
% % Forrit sem reiknar nálgun á f'(a) með Richardson útgiskun 
% % út frá miðsettum mismunakvótum 
% %   f'(a) ~ (f(a+h)-f(a-h))/(2h) 
% % Hér er h0 upphafsskrefalengd, epsilon er gefið skekkjumark 
% % og imax er hámarksfjöldi ítrana. 
% % Upphafsstillingar 
% h = h0; D = zeros(imax);  
% i = 2; mat = 2*epsilon; 
% D(1,1) = (f(a+h)-f(a-h))/(2*h); 
%     
% % Ítrum meðan ástæða er til 
% while (i <= imax $\&$ abs(mat) >= epsilon) 
%     h = h/2; 
%     D(i,1) = (f(a+h)-f(a-h))/(2*h); 
%     % Göngum á línu i 
%     for j=2:i 
%         mat = (D(i,j-1)-D(i-1,j-1))/(4**(j-1)-1); 
%         D(i,j) = D(i,j-1) + mat; 
%     end 
%     i = i+1; 
% end 
%    
% % Ath. að i endar einum hærra en maður býst við, 
% % pössum að skila réttu staki 
% R = D(i-1,i-1); 
% \end{verbatim}
% \end{frame}


\begin{frame}{6.3 Sýnidæmi} 
Látum $f(x)=x/\root 3 \of{x^2+4}$ og $a=-1$.  
Með því að byrja á $h=1$ og prentum út útgiskunartöfluna

\smallskip
\begin{center}
\begin{tabular}{l|llll}
$h$&$D(i,1)$&$D(i,2)$&$D(i,3)$&$D(i,4)$\\\hline
1.&   0.50000000 \\
0.5&   0.50564632  & 0.50752843\\
0.25&   0.50657385 &  0.50688303 &  0.50684000\\
0.125&   0.50676839 &  0.50683323 &  0.50682991 &  0.50682976
\end{tabular}
\end{center}

Niðustaðan er: $f'(-1)\approx   0.50682976$

með eftirámat á skekkju  $-1.6\cdot 10^{-7}$.  

\smallskip
Rétt gildi er $0.50682974129023$.
\end{frame}

%\end{document}

\section*{6.4 Töluleg heildun}
\begin{frame}{6.4 Töluleg heildun} 
Gerum ráð fyrir að $x_0,x_1, \ldots, x_n$ séu punktar á bilinu $[a,b]$
og að við þekkjum gildi $f$ í þessum punktum. Þá getum við fundið
brúunarmargliðuna $p_n$ gegnum punktana $(x_k,f(x_k))$ og skrifað 
\begin{equation*}
  f(x) = p_n(x) + r_n(x),
\end{equation*} 
þar sem leifin $r_n$ er gefin með
\begin{equation*}
  r_n(x) = f[x_0,\ldots,x_n,x](x-x_0)\cdots(x-x_n).
\end{equation*}
\pause
Nú er auðvelt að reikna heildi margliða, svo við nálgum heildi $f$ með
\begin{equation*}
  \int\limits_a^b f(x) dx \approx 
  I_n(f) := \int\limits_a^b p_n(x) dx
\end{equation*}
\pause
og skekkjan í þessari nálgun er gefin með
\begin{equation*}
  e_n = \int\limits_a^b r_n(x) dx.
\end{equation*}
Þessi aðferð er kölluð {\it Newton-Cotes-heildun}. 
\end{frame}


\begin{frame}{6.4 Newton-Cotes -heildun} 
Hugsum okkur að brúunarpunktarnir $x_0, \ldots, x_n$ séu ólíkir. Þá
getum við skrifað $p_n$ með Lagrange-margliðum 
\begin{equation*}
  p_n(x) = \sum\limits_{k=0}^n f(x_k) \ell_k(x),
  \quad
  \ell_k(x) = \prod\limits_{\stackrel{j=0}{j \not= k}}^n
  \frac{(x-x_j)}{(x_k-x_j)},
\end{equation*}
\pause
og þá er heildi $p_n$ jafnt
\begin{equation*}
  \int\limits_a^b p_n(x) dx = 
  \sum\limits_{k=0}^n f(x_k) A_k,
  \quad \text{þar sem} \quad
  A_k = \int\limits_a^b \ell_k(x) dx.
\end{equation*}
\pause
Athugið að gildi $A_k$ veltur aðeins á brúunarpunktunum $x_0, \ldots,
x_n$ en ekki gildum $f(x_k)$. Ef það á að heilda mörg föll yfir sama
bil er því hægt að reikna gildi $A_k$ í eitt skipti fyrir öll og
endurnýta þau svo. 
\end{frame}


\begin{frame}{6.4 Sýnidæmi} 
Metum heildi $f(x) = e^{-x}\cos(x)$ og $g(x) = \sin (\frac{x^2}{2})$
yfir bilið $[0,2]$ með að nota skiptipunktana $x_0 = 0$, $x_1 = 1$ og
$x_2 = 2$. Lagrange-margliðurnar sem við eiga eru 
\begin{equation*}
  \ell_0(x) = \frac{(x-1)(x-2)}{2}, \quad
  \ell_1(x) = -x(x-2), \quad
  \ell_2(x) = \frac{x(x-1)}{2}
\end{equation*}
\pause
svo við fáum að
\begin{gather*}
  A_0 = \frac{1}{2} \int\limits_0^2 (x-1)(x-2) dx = \frac{1}{3},
  \qquad
  A_1 = -\int\limits_0^2 x(x-2) dx = \frac{4}{3}, \\
  A_2 = \frac{1}{2} \int\limits_0^2 x(x-1) dx = \frac{1}{3}.
\end{gather*}
\end{frame}


\begin{frame}{6.4 Sýnidæmi frh.} 
Nú eru stuðlarnir fundnir og því fáum við 
\begin{align*}
  \int\limits_0^2 f(x) dx &\approx
  f(0)\frac{1}{3} + f(1)\frac{4}{3} + f(2)\frac{1}{3}\\
  &= \frac{1 + 4e^{-1}\cos(1) + e^{-2}\cos(2)}{3}
  \approx 0.59581
\end{align*}
og
\begin{align*}
  \int\limits_0^2 g(x) dx &\approx
  g(0)\frac{1}{3} + g(1)\frac{4}{3} + g(2)\frac{1}{3}\\
 & = \frac{4\sin(1/2) + \sin(2)}{3} 
  \approx 0.91972.
\end{align*}
\pause
Gildi heildanna eru
$\int\limits_0^2 f(x) dx \approx 0.58969$
og $\int\limits_0^2 g(x) dx \approx 0.99762$ með 5 réttum auka\-stöfum
svo nálgunargildin verða að teljast nokkuð góð miðað við hversu 
lítið fór í þau.
\end{frame}


\begin{frame}{6.4 Trapisuregla} 
Nú ætlum við að leiða út formúlur fyrir helstu reglum fyrir nálgun á
heildum.  Sú fyrsta er {\it trapisuregla}.

\pause
\smallskip
Veljum $x_0 = a$ og $x_1 = b$ sem skiptipunktana okkar. Þá er
graf $p_1$ línustrikið gegnum $(a,f(a))$ og $(b,f(b))$, 
\begin{equation*}
  p_1(x) = f(a) \ell_0(x) + f(b) \ell_1(x)
  = f(a)\frac{b-x}{b-a} + f(b) \frac{x-a}{b-a}
\end{equation*}
og vigtirnar eru
\begin{equation*}
  A_0 = \int\limits_a^b \ell_0(x) = \frac{b-a}{2} = A_1,
\end{equation*}
svo
\begin{equation*}
  \int\limits_a^b f(x) dx \approx 
  \frac{b-a}{2}\left(f(a)+f(b)\right).
\end{equation*}
Trapisureglan er kölluð þessu nafni því með henni nálgum við heildi
$f$ með flatarmáli trapisunnar sem hefur hornpunktana $(a,0)$,
$(b,0)$, $(b,f(b))$ og $(a,f(a))$. 
\end{frame}


\begin{frame}{6.4 Miðpunktsregla} 
Enn einfaldari er miðpunktsreglan, þá veljum við aðeins einn skiptipunkt, 
$x_0 = \frac{1}{2}(a+b)$, og brúunarmargliðan verður fastamargliðan 
$p_0(x) = f(x_0)$. Þá er
\begin{equation*}
  \int\limits_a^b f(x) dx \approx (b-a)f\left(\frac{a+b}{2}\right)
\end{equation*}
\end{frame}


\begin{frame}{6.4 Regla Simpsons} 
Nú veljum við þrjá skiptipunkta, $x_0 = a$, $x_1 = b$ og $x_2 =
\frac{1}{2}(a+b)$. Til einföldunar skulum við hliðra fallinu $f$
um miðpunkt bilsins $m=\tfrac{1}{2}(a+b)$.

\smallskip
Við skilgreinum $\alpha=\tfrac 12(b-a)$ og 
$g(x) = f\big(x+m\big)$

\pause
Þá hliðrast $a$, $m$ og $b$ yfir í $-\alpha$, $0$ og $\alpha$ og 
\begin{equation*}
  \int\limits_{-\alpha}^{\alpha} g(x) dx = 
  \int\limits_a^b f(x) dx.
\end{equation*}
\end{frame}


\begin{frame}{6.4 Regla Simpsons} 
Lagrange margliðurnar og vigtirnar eru
\begin{align*}
  l_0(x) &= \frac{(x-\alpha)x}{(-\alpha-\alpha)(-\alpha - 0)} 
  = \frac{(x-\alpha)x}{2\alpha^2} \\
  A_0 &= \int_{-\alpha}^{\alpha} l_0(x)\,dx = \frac{\alpha}{3} \\
  l_1(x) &= \frac{(x-(-\alpha))(x-0)}{(\alpha - ( -\alpha))(\alpha - 0)}
  = \frac{(x+\alpha)x}{2\alpha^2}\\
  A_1 &= \int_{-\alpha}^{\alpha} l_1(x)\,dx = \frac{\alpha}{3}\\
  l_2(x) &= \frac{(x-(\alpha))(x-\alpha)}{0-(-\alpha)(0-\alpha)}
  = \frac{(x+\alpha)(x-\alpha)}{-\alpha^2}\\
  A_2 &= \int_{\alpha}^{\alpha} l_2(x)\,dx = \frac{4\alpha}{3}
\end{align*}
\end{frame}


\begin{frame}{6.4 Regla Simpsons} 
Nálgunarformúlan verður þá
\begin{align*}
  \int_a^b f(x) \, dx = \int\limits_{-\alpha}^{\alpha} g(x) \, dx
  &\approx \frac{\alpha}{3}g(-\alpha) + \frac{\alpha}{3}g(\alpha) 
  + \frac{4\alpha}{3}g(0)\\
  &=(b-a)\left( \frac{1}{6}f(a) + \frac{4}{6}f
    \left( \frac{a+b}{2}\right) + \frac{1}{6} f(b)  \right)
\end{align*}
\end{frame}


\begin{frame}{6.4 Regla Simpsons aftur}
Ef við tökum brúunarmargliðu gegnum $a$, $b$ og $\frac{1}{2}(a+b)$ með
$\frac{1}{2}(a+b)$ tvöfaldan þá fáum við 3. stigs brúunarmargliðu 
\begin{equation*}
  p_3(x) = p_2(x) + g[-\alpha, \alpha, 0, 0](x+\alpha)(x-\alpha)x
\end{equation*} \pause
Heildið yfir seinni liðinn hægra megin er 0 því 
margliðan $(x+a)(x-a)x$ er oddstæð, \pause
en heildið yfir 
fyrri liðinn er 
$$
  \frac \alpha3(g(-\alpha) + 4g(0) + g(\alpha)).
$$
\pause
Út kemur því Simpson-regla.  
\end{frame}


\begin{frame}{6.4 Samsettu reglurnar} 
Þar sem Newton-Cotes heildun notar brúunarmargliður fylgja henni
nokkur vandamál. 

\pause
\smallskip
Ef okkur finnst nákvæmnin í nálguninni vera of lítil
getum við ekki búist við að hún batni við að fjölga skiptipunktum; þá
hækkar stig margliðunnar líklega sem orsakar sveiflukenndari
hegðun. 

\pause
\smallskip
Eins er ekki gott að halda sig við margliður af lægra stigi;
ef bilið sem á að heilda yfir er stórt væri mikil tilviljun að 1.,
2. eða 3. stigs brúunarmargliða nálgaði fallið vel á öllu bilinu. 
\end{frame}


\begin{frame}{6.4 Samsettu reglurnar} 
Lausnin á þessu vandamáli er í sama anda og fyrir splæsibrúun. \pause
Við veljum skiptingu 
\begin{equation*}
  a  =x_0 < x_1 < \ldots < x_n = b
\end{equation*}
á bilinu $[a,b]$. 

\pause
\smallskip
Um heildi gildir að
\begin{equation*}
  \int\limits_a^bf(x)\, dx = \sum\limits_{k=1}^n \ \ \int\limits_{x_{k-1}}^{x_k} f(x) \, dx
\end{equation*}
svo við getum nálgað heildi $f$ á sérhverju litlu hlutbili
$[x_{k-1},x_k]$ með að heilda brúunarmargliðu af lágu stigi og lagt
öll gildin saman til að fá nálgun á heildi $f$ yfir allt bilið. 

\pause
\smallskip
Þegar ákveðin regla er notuð til að nálga heildi $f$ á sérhverju
hlutbili er þetta kölluð {\it samsetta} útgáfa reglunnar. Einfalt er að leiða
út samsettar útgáfur reglanna að ofan. 
\end{frame}


\begin{frame}{6.4 Samsetta trapisureglan} 
Á sérhverju hlutbili er
\begin{equation*}
  \int\limits_{x_{k-1}}^{x_k} f(x) \, dx
  \approx
  \frac{x_k-x_{k-1}}{2}(f(x_{k-1}) + f(x_k))
\end{equation*}
svo
\begin{equation*}
  \int\limits_a^b f(x) \, dx
  \approx
  \sum\limits_{k=1}^n \frac{x_k-x_{k-1}}{2}(f(x_{k-1}) + f(x_k)).
\end{equation*}

\pause
\smallskip
Ef öll hlutbilin eru jafn löng og $h = x_k-x_{k-1}$, \pause
þá fæst
\begin{multline*}
  \int\limits_a^b f(x) \, dx \\
  \approx 
  h\left( \frac{1}{2}f(a) + f(a+h) + f(a+2h) 
    + \cdots + f(a+(n-1)h) + \frac{1}{2}f(b) \right).
\end{multline*}\end{frame}


\begin{frame}{6.4 Samsetta miðpunktsreglan} 
Fljótséð er að
\begin{equation*}
  \int\limits_a^b f(x) \, dx
  \approx
  \sum\limits_{k=1}^n (x_k-x_{k-1})f
  \left(
    \frac{x_{k-1}+x_k}{2}
  \right)
\end{equation*}

\pause
\smallskip
Ef öll hlutbilin eru jafn löng verður formúlan
\begin{equation*}
  \int\limits_a^b f(x) \, dx
  \approx
  h \sum\limits_{k=1}^n f \left(\frac{x_{k-1}+x_k}{2}\right)
\end{equation*}
\end{frame}


\begin{frame}{6.4 Samsetta Simpson}
Hér er venjan að velja $2n+1$ jafndreifða skiptipunkta og fá $n$ jafn
stór hlutbil. Þá er $h = \frac{b-a}{2n}$, $x_k = a + kh$ fyrir $k =
0,\ldots,2n$ og hlutbilin eru $[x_{2k-2},x_{2k}]$ fyrir $k = 1,
\ldots, n$. 

\pause
\smallskip
Á hverju hlutbili er 
\begin{equation*}
  \int\limits_{t_{2k-2}}^{t_{2k}} f(x) \, dx
  \approx
  2h \left(
    \frac{1}{6} f(x_{2k-2}) + \frac{4}{6} f(x_{2k-1}) 
    + \frac{1}{6} f(x_{2k})
  \right)
\end{equation*}
\pause
svo að
{\small 
\begin{align*}
  \int\limits_a^b f(x) \, dx
  \approx &
  \sum\limits_{k=1}^n
  \bigg(
    \frac{h}{3}
    \Big(
      f(x_{2k-2}) + 4f(x_{2k-1}) + f(x_{2k})
    \Big)
  \bigg) \\
  = &
  \frac{h}{3}
  \Big( 
    f(a) + 4f(a+h) + 2f(a+2h)+ 4f(a+3h) + 2f(a+4h) \\
    &+ \cdots + 2f(a+(2n-2)h) + 4f(a+(2n-1)h) + f(b).
  \Big)
\end{align*}}
\end{frame}


\begin{frame}{6.4 Skekkjumat}
Rifjum upp grunnhugmyndina að baki nálgunarformúlunum. Við veljum
brúunarpunkta $x_0, \ldots, x_n$ í $[a,b]$, látum $p_n$ vera
tilsvarandi brúunarmargliðu og skrifum 
\begin{equation*}
  f(x) = p_n(x) + r_n(x)
\end{equation*}
\pause
þar sem $r_n(x) = f[x_0, \ldots , x_n, x](x-x_0) \cdots (x-x_n)$. Þá er nálgunin
\begin{equation*}
  \int_a^b f(x)\,dx \approx \int_a^b p_n(x)\,dx
\end{equation*}
\pause
með skekkjuna
\begin{equation*}
  \int_a^b r_n(x)\,dx
\end{equation*}
Nú viljum við meta skekkjuheildið.
\end{frame}

\begin{frame}{6.4 Meðalgildissetningin fyrir heildi}
 Við skekkjumatið í þessum kafla munum við þurfa að nota eftirafarandi 
 setningu nokkrum sinnum.
\begin{block}{Setning (Meðalgildissetningin fyrir heildi):}
 Ef $G:[a,b] \to \R$ er samfellt fall og $\phi$ er heildanlegt fall sem 
 skiptir ekki um formerki á bilinu $[a,b]$ þá er til tala $\eta \in [a,b]$
 þannig að 
 $$
  \int_a^b G(x)\phi(x)\, dx = G(\eta) \int_a^b \phi(x)\, dx.
 $$
\end{block}

\end{frame}


\begin{frame}{6.4 Trapisuregla} 
\begin{equation*}
  r_1(x) = f[-\alpha, \alpha, x](x+\alpha)(x-\alpha)
\end{equation*}
\pause
Athugum að 
\begin{equation*}
  (x+\alpha)(x-\alpha) = (x^2 - \alpha^2)
\end{equation*}
skiptir ekki um formerki á bilinu $]-\alpha, \alpha[$. \pause
Þá gefur meðalgildissetningin fyrir heildi að til 
er $\eta \in [a,b]$ þannig að\pause
\begin{align*}
  \int_a^b r_1(x)\,dx 
  &= f[-\alpha, \alpha, \eta]
  \int_{-\alpha}^{\alpha}(x^2 - \alpha^2)\,dx\\
  &= \frac{f''(\xi)}{2!} \left( - \frac{4}{3}\alpha^3 \right)\\
  &= \frac{-f''(\xi)}{2!}\frac{(b-a)^3}{6}, \qquad \xi \in [a,b]
\end{align*}

\pause
\smallskip
Niðurstaða:
\begin{equation*}
  \int_a^b f(x)\,dx = (b-a)
  \left( \frac{1}{2} f(a) + \frac{1}{2}f(b) \right) 
  - \frac{1}{12} f''(\xi)(b-a)^3
\end{equation*}
\end{frame}


\begin{frame}{6.4 Skekkjumat í samsettu reglunni} 
Ef við lítum á samsettu trapisuregluna með jafna skiptingu þar sem 
hlutbilin eru $[x_i,
x_{i+1}]$, þá fáum við skekkjuna 
\begin{equation*}
  - \frac{h^3}{12}f''(\xi_i), \qquad \xi_i \in [x_i, x_{i+1}]
\end{equation*}
\pause
Ef við leggjum saman og beitum milligildissetningunni, þá fáum við
\begin{equation*}
  \int_a^b f(x)\,dx = T(h) - \frac{h^2}{12}(b-a)f''(\xi), \qquad 
  \xi \in [a,b]
\end{equation*}
að því gefnu að $f\in C^2 [a,b]$.

\smallskip
Ath: Hér er $T(h)$ útkoman úr samsettu Trapisureglunni með 
jafna skiptingu $h = \frac{b-a}n$.
\end{frame}


\begin{frame}{6.4 Skekkja í miðpunktsreglu} 
Til einföldunar skoðum við bilið $[-\alpha,\alpha]$.
Veljum miðpunktinn tvöfaldan
\begin{align*}
  &p_1(x) = f(0) + f'(0)x\\
  &r_1(x) = f[0,0,x]x^2
\end{align*}
Athugum að heildið af $f'(0)x$ yfir $[-\alpha,\alpha]$ er 0. \pause
Nú skiptir $x^2$ ekki um formerki og því gefur meðalgildisreglan 
fyrir heildi að til er $\eta \in [-\alpha,\alpha]$ þannig að
\begin{align*}
  \int_a^b r_1(x)\,dx 
  &= \int_{-\alpha}^{\alpha} f[0,0,x]x^2 \,dx\\
  &= f[0,0,\eta]\int_{-\alpha}^\alpha x^2\,dx\\
  &= \frac{f''(\xi)}{2!}2\frac{\alpha^3}{3}\\
  &= \frac{(b-a)^3}{24}\cdot f''(\xi)
\end{align*}
Þar sem $\xi$ fæst úr skekkjumatinu fyrir brúunarmargliður (kafli 5).
\end{frame}


\begin{frame}{6.4 Skekkja í samsettu miðpunktsreglu}
Fyrir hvert bil fáum við skekkjulið:
\begin{equation*}
  \frac{h^3}{24}\cdot f''(\xi_i)
\end{equation*}
\pause
Leggjum saman skekkjuliðina og beitum milligildissetningunni, þá fæst
að til er $\xi$ þannig að:
\begin{equation*}
  \int_a^b f(x)\,dx = h \sum_{i=1}^n 
  f\left(a+ (i - \frac{1}{2})h\right) + \frac{b-a}{24}f''(\xi)h^2
\end{equation*} 
\end{frame}


\begin{frame}{6.4 Skekkja í reglu Simpsons} 
\begin{equation*}
  \int_a^b f(x)\,dx \approx (b-a) 
  \left( 
    \frac{1}{6}f(a) + \frac{4}{6}f
    \left( \frac{1}{2}(a+b) \right) + \frac{1}{6}f(b)
  \right)
\end{equation*}
\pause
Leiddum út þessa formúlu með því að taka brúunarmargliðu $p_3(x)$ með
punktana $-\alpha, \alpha, 0, 0$. Skekkjan er 
\begin{equation*}
  f(x) - p_3(x) = f[-\alpha, \alpha, 0, 0, x]
  (x+\alpha)(x-\alpha)x^2
\end{equation*}
\pause
þar með er skekkjan í formúlu Simpsons:
\begin{equation*}
  \int_{-\alpha}^{\alpha}f[-\alpha, \alpha, 0, 0, x]
  (x+\alpha)(x-\alpha)x^2 \,dx
\end{equation*}
\end{frame}


\begin{frame}{6.4 Skekkja í reglu Simpsons, frh.} 
Fallið $x\mapsto (x+\alpha)(x-\alpha)x^2 = (x^2 - \alpha^2)x^2$ er
$\leq 0$ á $[-\alpha, \alpha]$. Þar með gefur meðalgildissetningin
fyrir heildi að til er $\eta \in [-\alpha, \alpha]$ þannig að skekkjan
er 
\begin{multline*}
  f[-\alpha, \alpha, 0, 0, \eta]
  \int_{-\alpha}^{\alpha}(x^2 - \alpha^2)x^2 \,dx \\
  = \frac{f^{(4)}(\xi)}{4!}\cdot \frac{(-4)}{15}\cdot \alpha^5
  = \frac{-f^{(4)}(\xi)}{90}\left(\frac{b-a}{2}\right)^5, \qquad 
  \xi \in [a,b]
\end{multline*}
Þar sem $\xi$ fæst úr
skekkjumatinu fyrir Newton aðferðina (glæra 5.55).
\end{frame}


\begin{frame}{6.4 Skekkja samsettu Simpsonreglu}
Skiptum $[a,b]$ í $n$ jafnlöng bil og látum $h$ vera helming
hlutbillengdarinnar,
\begin{equation*}
  h = \frac{(b-a)}{2n}.
\end{equation*}
\pause
Þá er
{\small
\begin{align*}
  \int\limits_a^b f(x) \, dx
  \approx &
  \sum\limits_{k=1}^n
  \bigg(
    \frac{h}{3}
    \Big(
      f(x_{2k-2}) + 4f(x_{2k-1}) + f(x_{2k})
    \Big)
  \bigg) \\
  = &
  \frac{h}{3}
  \Big( 
    f(a) + 4f(a+h) + 2f(a+2h)+ 4f(a+3h) + 2f(a+4h) \\
    &+ \cdots + 2f(a+(2n-2)h) + 4f(a+(2n-1)h) + f(b)
  \Big)
\end{align*}}
\end{frame}


\begin{frame}{6.4 Skekkjumat í samsettu Simpsonreglu}
Ef við beitum skekkjumatinu á sérhvert bilanna þá fáum við 
\begin{equation*}
  \frac{-f^{(4)}(\xi_i)}{90}h^5
\end{equation*}
sem skekkju með $\xi_i \in [x_i, x_i+1]$. Heildarskekkjan verður
\begin{equation*}
  -\sum_{i=1}^n \frac{f^{(4)}(\xi_i)}{90}h^5 
  = \frac{-h^5}{90}\cdot \sum_{i=1}^n f^{(4)}(\xi_i)
\end{equation*}
\pause
Nú gefur meðalgildisreglan að til er $\xi \in [a,b]$ þannig að
\begin{equation*}
  f^{(4)}(\xi) = \frac{1}{n} \sum_{i=1}^n f^{(4)}(\xi_i)
\end{equation*}
\pause
Nú er $nh = \frac{(b-a)}{2}$ þar með er skekkjan:
\begin{equation*}
  \frac{-h^5}{90}\cdot nf^{(4)}(\xi) 
  = \frac{-(b-a)}{180}f^{(4)}(\xi)\cdot h^4
\end{equation*}
\end{frame}

\begin{frame}{6.4 Skekkjumat á samsettu Simpsonreglunni: Niðurstaða}
\pause
Ef við táknum útkomuna úr samsettu Simpsonsreglunni fyrir 
$h=\frac{b-a}{2n}$ með $S(h)$ þá fæst að til er
$\xi \in [a,b]$ þannig að
\begin{equation*}
  \int_a^b f(x)\,dx = S(h) - \frac{(b-a)}{180}f^{(4)}(\xi)h^4
\end{equation*}
\end{frame}

\section*{6.7 Romberg-útgiskun}
\begin{frame}{6.7 Romberg-útgiskun} 
Á sama hátt og við gátum bætt nálgun okkar á afleiðu falls með að nota
Richardson útgiskun getum við bætt nálgun á heildi. 

\pause
\smallskip
Aðferðin virkar í
aðalatriðum eins fyrir heildi og afleiður, en til að fá sem bestar
upplýsingar um samleitni hennar skulum við leiða út formúluna fyrir
trapisureglunni aftur. 
\end{frame}


\begin{frame}{6.7 Euler-Maclauren-formúlan} 
Fyrir samfellt fall $f : [0,1] \to \mathbb R$ sem er $2n$-sinnum
samfellt deildanlegt gildir Euler-Maclauren formúlan 
\begin{align*}
  \int\limits_0^1 f(t) \, dt 
  =&  \frac{1}{2}\left( f(0) + f(1) \right) 
  + \sum\limits_{k=1}^{n-1} A_{2k}
  \left( f^{(2k-1)}(0) - f^{(2k-1)}(1)\right) \\
  & - A_{2n}f^{(2n)}(\xi), \qquad \xi \in [0,1]
\end{align*}
\pause
Hér eru stuðlarnir $A_k$ þannig að $k!A_k$ verði Bernoulli-talan númer
$k$. Þessar tölur eru stuðlar í veldaröðinni 
\begin{equation*}
  \frac{x}{e^x -1} = \sum\limits_{k=0}^{\infty}A_kx^k
\end{equation*}

\pause
(Það þarf að hafa töluvert fyrir því að sanna þessa formúlu)
\end{frame}


\begin{frame}{6.7 Afleiðing af Euler-Maclaurin-formúlu}
Látum nú $f : [a,b] \to \mathbb R$ vera $2n$-sinnum samfellt
deildanlegt fall. \pause
Ef við búum til skiptingu $a= x_0 < x_1 < \cdots <
x_n = b$ með jöfn hlutbil $h = x_{i+1} - x_i$ og beitum síðan
Euler-Maclauren formúlunni á $g(t) = f(x_i + ht)$ fæst 
\begin{align*}
   \int_{x_i}^{x_{i+1}} f(x)\,dx 
  = & h\int_0^1 \underbrace{f(x_i + ht)}_{g(t)}\,dt \\
  = & {\color{blue} h \left( \frac{1}{2}f(x_i) + \frac{1}{2}f(x_{i+1})\right) }\\
   & +    \sum_{k=1}^{n-1}A_{2k}h^{2k}\left( f^{(2k-1)}(x_i) -
    f^{(2k-1)}(x_{i+1}) \right) \\
    & - A_{2n}h^{2n+1}f^{(2n)}(\xi_i), 
\end{align*}
þar sem $\xi_i \in [x_i, x_{i+1}]$.
\end{frame}

\begin{frame}{6.7 Afleiðing af Euler-Maclaurin-formúlu}

Nú innleiðum við
\begin{align*}
  T(h) 
  &:= \sum_{i=0}^{n-1}
  {\color{blue} h \left( \frac{1}{2} f(x_i) + 
  \frac{1}{2}f(x_{i+1}) \right)}\\
  &= h\left( \frac{1}{2}f(a) + f(a+h) 
    + \cdots + f(a+(n-1)h) + \frac{1}{2}f(a+nh)\right)
\end{align*}
\pause
og fáum síðan:
\begin{align*}
  \int\limits_a^b f(x)\, dx 
  = & T(h) + \sum_{k=1}^{n-1}A_{2k}h^{2k} 
  \left( f^{(2k-1)}(a) - f^{(2k-1)}(b) \right) \\
  & - A_{2n}h^{2n+1} \sum_{i=0}^{n-1} f^{(2n)}(\xi_i)
\end{align*}
\end{frame}


\begin{frame}{6.7 Afleiðing af Euler-Maclaurin} 
Nú gefur milligildissetningin að til er $\xi \in [a,b]$ þannig að
\begin{equation*}
  \frac{1}{n} \sum\limits_{k=0}^{n-1} f^{(2n)}(\xi_i)
  = f^{(2n)}(\xi)
\end{equation*}

Notum okkur nú að $nh = b-a$ og fáum að
\begin{align*}
  \int\limits_a^b f(x) \, dx 
  = & T(h) + \sum_{k=1}^{n-1}A_{2k}h^{2k} 
  \left( f^{(2k-1)}(a) - f^{(2k-1)}(b) \right) \\
  & - A_{2n} h^{2n}(b-a)f^{(2n)}(\xi).
\end{align*}

\pause
Niðurstaðan er að samsetta trapisureglan er
\begin{equation*}
  \int\limits_a^b f(x) \, dx 
  = T(h) + c_2h^2 + c_4h^4 + \cdots + c_{2m-2}h^{2m-2} 
  + c_{2n}h^{2m}f^{(2m)}(\xi)
\end{equation*}
\end{frame}


\begin{frame}{6.7 Ítrekun á samsettu trapisureglunni með helmingun} 
Hugsum okkur nú að við viljum reikna út $T(h_j)$ fyrir $h_j =(b-a)/
2^j$, $j = 1,2,\ldots$ og að við viljum nýta öll fallgildi í
$T(h_{j-1})$  til að reikna út $T(h_j)$. Rakningarformúlan er 
\begin{equation*}
  T(h_j) = \frac{1}{2} T(h_{j-1}) + h_j \sum_{k=1}^{2^{j-1}} f(a+(2k-1)h_j)
\end{equation*}
Athugið að hér er bilinu $[a,b]$ skipt í $2^j$ hlutbil.
\end{frame}


\begin{frame}{6.7 Reikniritið fyrir Romberg-heildun} 
Romberg-heildun er hugsuð nákvæmlega eins og Richardson-útgiskunin:
Við reiknum út línu fyrir línu í töflunni: 
\begin{equation*}
  \begin{array}{cccccc}
    i\\
    1 & R(1,1)\\
    2 & R(2,1) & R(2,2)\\
    3 & R(3,1) & R(3,2) & R(3,3)\\
    4 & R(4,1) & R(4,2) & R(4,3) & R(4,4)\\
    \vdots & \vdots & \vdots & \vdots & \vdots & \ddots
  \end{array}
\end{equation*}
þar sem
\begin{align*}
  &R(i,1) = T(h_i) \qquad i = 1,2,\ldots\\
  &R(i,j) = \frac{4^{j-1} R(i,j-1) - R(i-1,j-1)}{4^{j-1} - 1}.
\end{align*}
Með þessu fæst 
$\int\limits_a^b f(x)\, dx = R(k,k) + O(h_k^{2k})$,
þar sem $k$ er síðasta línan sem við reiknum í töflunni að ofan.
\end{frame}

\begin{frame}{6.7 Skekkjumat í Romberg heildun} 
Hægt er að nota síðustu viðbót sem eftirámat fyrir skekkjuna, þetta mat er
\begin{equation*}
  \frac{1}{4^{j-1}-1}\left( R(i,j-1) - R(i-1,j-1)\right)
\end{equation*}
þegar þessi stærð er komin niður fyrir fyrirfram gefin skekkjumörk er
hætt. 

\medskip\pause

Athugið að það er ekki nauðsynlegt að hafa $h_1$ sem allt bilið $[a,b]$, það er ekkert sem 
kemur í veg fyrir það að við byrjum með $h_1 = \frac{b-a}{m}$, og helmingum svo;
$h_2 = \frac{b-a}{2m}$, $h_3 = \frac{b-a}{4m}$, $\ldots$.\\
Almennt er þá $h_j=\frac{b-a}{2^{j-1}m}$.
\end{frame}


\section*{Fræðilegar spurningar}
\begin{frame}{Kafli 6: Fræðilegar spurningar}
  \begin{enumerate}
  \item  Hver er meginhugmyndin í tölulegri deildun og heildun?
  \item  Hvað eru {\it frammismunur} og {\it bakmismunur} til þess að
    nálga afleiðu?
  \item  Hvernig er {\it miðsettur mismunakvóti} fyrir fyrsta stigs
    afleiðu skilgreindur og hver 
    er skekkjan í nálgun á afleiðu falls með honum?
  \item  Hvernig er {\it miðsettur mismunakvóti} fyrir annars stigs
    afleiðu skilgreindur og hvernig er skekkjan í nálgun á annarri
    afleiðu með honum?
  \item  Hvernig eru  brúnunarmargliður notaðar til þess að
    reikna út afleiðu falls $f$ í punkti $a$ og hver er skekkjan í
    slíkri nálgun?
  \item  Lýsið fyrsta skrefinu i Richardson-útgiskun þar sem 
formúlan $f'(a)=R_0(h)+a_2h^2+a_4h^4+O(h^6)$ er endurbætt þannig að út
komi skekkja sem er $O(h^4)$.
  \item  Lýsið Richardson-útgiskunartöflunni. 
  \item  Hvaða skekkjumat er notað í Richardson-útgiskun?
  \end{enumerate}
\end{frame}


\begin{frame}{Kafli 6: Fræðilegar spurningar}
  \begin{enumerate}
  \item [9.] Hvernig er almenna aðferðin sem notar brúunarmargliður til þess
    að nálga heildi og nefnd er Newton-Cotes-heildun og hvernig er
    skekkjuformúlan í henni? 
  \item [10.] Hvernig er trapisuregla til þess að nálga heildi og
   aðferðarskekkja hennar?
  \item [11.] Hvernig er miðpunktsregla til þess að nálga heildi og
   aðferðarskekkja hennar? 
  \item [12.] Hvernig er Simpson-regla til þess að nálga heildi og
    aðferðarskekkja hennar?
  \item [13.] Hvernig er samsetta trapisureglan og 
    aðferðarskekkja hennar?
  \item [14.] Hvernig er samsetta miðpunktsreglan og 
 aðferðarskekkja hennar? 
  \item [15.]  Hvernig er samsetta Simpson-reglan og 
aðferðarskekkja hennar?
  \item [16.] Hvernig er rakningarformúla fyrir samsettu trapisureglunni?
  \item [17.] Lýsið reikniritinu fyrir Romberg-heildun.
  \item [18.] Hver er skekkjan í eftirámatinu í Romberg-heildun?
   \end{enumerate}
\end{frame}

\end{document}